%%==================================================
%% diss.tex for SJTU Master Thesis
%% based on CASthesis
%% modified by wei.jianwen@gmail.com
%% version: 0.3a
%% Encoding: UTF-8
%% last update: Dec 5th, 2010
%%==================================================

% 字号选项: c5size 五号(默认) cs4size 小四
% 双面打印(注意字号设置)
\documentclass[cs4size, a4paper, twoside]{sjtuthesis} 
% 单面打印(注意字号设置)
% \documentclass[cs4size, a4paper, oneside, openany]{sjtuthesis} 


% \usepackage[sectionbib]{chapterbib}%每章都用参考文献
\usepackage{semantic}
\usepackage[all]{xy}
\usepackage{tikz}
\usepackage{stmaryrd}
\usepackage{algorithm,algpseudocode}
\newboolean{DOIT}
\setboolean{DOIT}{false}%编译某些只想自己看的内容,编译true,否则false

%% 行距缩放因子(x倍字号)
\renewcommand{\baselinestretch}{1.3}
\newcommand{\enc}[1]{\llbracket #1\rrbracket}
\newcommand{\ddef}{\overset{\textrm{def}}{=}}
\newcommand{\act}[1]{{\xrightarrow{#1}{}}}
\newcommand{\Act}[1]{{\stackrel{#1}{\Longrightarrow}} }
\newcommand{\rand}[1]{\marginpar{\raggedright\footnotesize #1}}
\newcommand{\todo}[1]{\rand{\textbf{Todo: }#1}}
\newcommand{\Exp}{{\approx^{\triangleleft}_{\pi}}}
\newcommand{\Cvg}{{\approx^{\triangleright}_{\pi}}}
\newcommand{\FS}{\textsf{FS}}
\newcommand{\BPP}{\textsf{BPP}}
\newcommand{\nBPP}{\textsf{nBPP}}
\newcommand{\PDA}{\textsf{PDA}}
\newcommand{\nPDA}{\textsf{nPDA}}
\newcommand{\BPA}{\textsf{BPA}}
\newcommand{\nBPA}{\textsf{nBPA}}
\newcommand{\PA}{\textsf{PA}}
\newcommand{\nPA}{\textsf{nPA}}
\newcommand{\tnPA}{\textsf{tnPA}}

%\newtheorem{definitions}{Definition}[section]

% \newcommand{\nBPA}{$\textbf{nBPA}$ } \newcommand{\BPA}{$\textbf{BPA}$ }
% \newcommand{\nBPP}{$\textbf{nBPP}$ } \newcommand{\BPP}{$\textbf{BPP}$ }
% \newcommand{\nPDA}{$\textbf{nPDA}$ } \newcommand{\PDA}{$\textbf{PDA}$ }
% \newcommand{\nPN}{$\textbf{nPN}$ } \newcommand{\PN}{$\textbf{PN}$ }
% \newcommand{\nPA}{$\textbf{nPA}$ } \newcommand{\PA}{$\textbf{PA}$ }
% \newcommand{\PRS}{$\textbf{PRS}$ } \newcommand{\FS}{$\textbf{FS}$ }
% \newcommand{\PAD}{$\textbf{PAD}$ } \newcommand{\PAN}{$\textbf{PAN}$ }
\newcommand{\beq}{\simeq} \newcommand{\weq}{\approx} \newcommand{\seq}{\sim}
\newcommand{\para}{\,|\,}
% 设置图形文件的搜索路径
\graphicspath{{figure/}{figures/}{logo/}{logos/}{graph/}{graphs}}

%%========================================
%% 在sjtuthesis.cls中定义的有用命令
%%========================================
% \cndash 中文破折号
% 数学常量
% \me 对数常数e
% \mi 虚数单位i
% \mj 虚数单位j
% \dif 直立的微分算符d为直立体。
% 可伸长的数学箭头、等号
% \myRightarrow{}{}
% \myLeftarrow{}{}
% \myBioarrow{}{}
% \myLongEqual{}{}
% 参考文献
% \upcite{} 上标引用
%%========================================


\begin{document}

%%%%%%%%%%%%%%%%%%%%%%%%%%%%%% 
%% 封面
%%%%%%%%%%%%%%%%%%%%%%%%%%%%%% 

% 中文封面内容(关注内容而不是形式)
\title{进程重写系统的有限性问题研究}
\author{杨\quad{}非}
\advisor{傅育熙教授}
\degree{硕士}
\defenddate{2014年1月16日}
\school{上海交通大学}
\institute{电子信息与电气工程学院}
\studentnumber{1110339027}
\major{计算机科学与技术}

% 英文封面内容(关注内容而不是表现形式)
\englishtitle{Regularity Problems of Process Rewrite Systems}
\englishauthor{\textsc{Fei Yang}}
\englishadvisor{Prof. \textsc{Yuxi Fu}}
\englishschool{Shanghai Jiao Tong University}
\englishinstitute{\textsc{Depart of Computer Science And Engineering} \\
  \textsc{Shanghai Jiao Tong University} \\
  \textsc{Shanghai, P.R.China}}
\englishdegree{Master}
\englishmajor{Computer Science}
\englishdate{Jan. 16th, 2014}

% 封面
\maketitle

% 英文封面
\makeenglishtitle

% 论文原创性声明和使用授权
\makeDeclareOriginal
\makeDeclareAuthorization

%%%%%%%%%%%%%%%%%%%%%%%%%%%%%% 
%% 前言
%%%%%%%%%%%%%%%%%%%%%%%%%%%%%% 
\frontmatter

% 摘要
%%==================================================
%% abstract.tex for SJTU Master Thesis
%% based on CASthesis
%% modified by wei.jianwen@gmail.com
%% version: 0.3a
%% Encoding: UTF-8
%% last update: Dec 5th, 2010
%%==================================================

\begin{abstract}

  上海交通大学是我国历史最悠久的高等学府之一,是教育部直属、教育部与上海市共建的全国重点大学,是国家 “七五”、“八五”重点建设和“211工程”、“985工程”的首批建设高校。经过115年的不懈努力,上海交通大学已经成为一所“综合性、研究型、国际化”的国内一流、国际知名大学,并正在向世界一流大学稳步迈进。 

 十九世纪末,甲午战败,民族危难。中国近代著名实业家、教育家盛宣怀和一批有识之士秉持“自强首在储才,储才必先兴学”的信念,于1896年在上海创办了交通大学的前身——南洋公学。建校伊始,学校即坚持“求实学,务实业”的宗旨,以培养“第一等人才”为教育目标,精勤进取,笃行不倦,在二十世纪二三十年代已成为国内著名的高等学府,被誉为“东方MIT”。抗战时期,广大师生历尽艰难,移转租界,内迁重庆,坚持办学,不少学生投笔从戎,浴血沙场。解放前夕,广大师生积极投身民主革命,学校被誉为“民主堡垒”。

 新中国成立初期,为配合国家经济建设的需要,学校调整出相当一部分优势专业、师资设备,支持国内兄弟院校的发展。五十年代中期,学校又响应国家建设大西北的号召,根据国务院决定,部分迁往西安,分为交通大学上海部分和西安部分。1959年3月两部分同时被列为全国重点大学,7月经国务院批准分别独立建制,交通大学上海部分启用“上海交通大学”校名。历经西迁、两地办学、独立办学等变迁,为构建新中国的高等教育体系,促进社会主义建设做出了重要贡献。六七十年代,学校先后归属国防科工委和六机部领导,积极投身国防人才培养和国防科研,为“两弹一星”和国防现代化做出了巨大贡献。

   改革开放以来,学校以“敢为天下先”的精神,大胆推进改革:率先组成教授代表团访问美国,率先实行校内管理体制改革,率先接受海外友人巨资捐赠等,有力地推动了学校的教学科研改革。1984年,邓小平同志亲切接见了学校领导和师生代表,对学校的各项改革给予了充分肯定。在国家和上海市的大力支持下,学校以“上水平、创一流”为目标,以学科建设为龙头,先后恢复和兴建了理科、管理学科、生命学科、法学和人文学科等。1999年,上海农学院并入;2005年,与上海第二医科大学强强合并。至此,学校完成了综合性大学的学科布局。近年来,通过国家“985工程”和“211工程”的建设,学校高层次人才日渐汇聚,科研实力快速提升,实现了向研究型大学的转变。与此同时,学校通过与美国密西根大学等世界一流大学的合作办学,实施国际化战略取得重要突破。1985年开始闵行校区建设,历经20多年,已基本建设成设施完善,环境优美的现代化大学校园,并已完成了办学重心向闵行校区的转移。学校现有徐汇、闵行、法华、七宝和重庆南路(卢湾)5个校区,总占地面积4840亩。通过一系列的改革和建设,学校的各项办学指标大幅度上升,实现了跨越式发展,整体实力显著增强,为建设世界一流大学奠定了坚实的基础。

  交通大学始终把人才培养作为办学的根本任务。一百多年来,学校为国家和社会培养了20余万各类优秀人才,包括一批杰出的政治家、科学家、社会活动家、实业家、工程技术专家和医学专家,如江泽民、陆定一、丁关根、汪道涵、钱学森、吴文俊、徐光宪、张光斗、黄炎培、邵力子、李叔同、蔡锷、邹韬奋、陈敏章、王振义、陈竺等。在中国科学院、中国工程院院士中,有200余位交大校友;在国家23位“两弹一星”功臣中,有6位交大校友;在18位国家最高科学技术奖获得者中,有3位来自交大。交大创造了中国近现代发展史上的诸多“第一”:中国最早的内燃机、最早的电机、最早的中文打字机等;新中国第一艘万吨轮、第一艘核潜艇、第一艘气垫船、第一艘水翼艇、自主设计的第一代战斗机、第一枚运载火箭、第一颗人造卫星、第一例心脏二尖瓣分离术、第一例成功移植同种原位肝手术、第一例成功抢救大面积烧伤病人手术等,都凝聚着交大师生和校友的心血智慧。改革开放以来,一批年轻的校友已在世界各地、各行各业崭露头角。

 截至2011年12月31日,学校共有24个学院/直属系(另有继续教育学院、技术学院和国际教育学院),19个直属单位,12家附属医院,全日制本科生16802人、研究生24495人(其中博士研究生5059人);有专任教师2979名,其中教授835名;中国科学院院士15名,中国工程院院士20名,中组部“千人计划”49名,“长江学者”95名,国家杰出青年基金获得者80名,国家重点基础研究发展计划(973计划)首席科学家24名,国家重大科学研究计划首席科学家9名,国家基金委创新研究群体6个,教育部创新团队17个。

  学校现有本科专业68个,涵盖经济学、法学、文学、理学、工学、农学、医学、管理学和艺术等九个学科门类;拥有国家级教学及人才培养基地7个,国家级校外实践教育基地5个,国家级实验教学示范中心5个,上海市实验教学示范中心4个;有国家级教学团队8个,上海市教学团队15个;有国家级教学名师7人,上海市教学名师35人;有国家级精品课程46门,上海市精品课程117门;有国家级双语示范课程7门;2001、2005和2009年,作为第一完成单位,共获得国家级教学成果37项、上海市教学成果157项。

  \keywords{\large 上海交大 \quad 饮水思源 \quad 爱国荣校}
\end{abstract}

\begin{englishabstract}

An imperial edict issued in 1896 by Emperor Guangxu, established Nanyang Public School in Shanghai. The normal school, school of foreign studies, middle school and a high school were established. Sheng Xuanhuai, the person responsible for proposing the idea to the emperor, became the first president and is regarded as the founder of the university.

During the 1930s, the university gained a reputation of nurturing top engineers. After the foundation of People's Republic, some faculties were transferred to other universities. A significant amount of its faculty were sent in 1956, by the national government, to Xi'an to help build up Xi'an Jiao Tong University in western China. Afterwards, the school was officially renamed Shanghai Jiao Tong University.

Since the reform and opening up policy in China, SJTU has taken the lead in management reform of institutions for higher education, regaining its vigor and vitality with an unprecedented momentum of growth. SJTU includes five beautiful campuses, Xuhui, Minhang, Luwan Qibao, and Fahua, taking up an area of about 3,225,833 m2. A number of disciplines have been advancing towards the top echelon internationally, and a batch of burgeoning branches of learning have taken an important position domestically.

Today SJTU has 31 schools (departments), 63 undergraduate programs, 250 masters-degree programs, 203 Ph.D. programs, 28 post-doctorate programs, and 11 state key laboratories and national engineering research centers.

SJTU boasts a large number of famous scientists and professors, including 35 academics of the Academy of Sciences and Academy of Engineering, 95 accredited professors and chair professors of the "Cheung Kong Scholars Program" and more than 2,000 professors and associate professors.

Its total enrollment of students amounts to 35,929, of which 1,564 are international students. There are 16,802 undergraduates, and 17,563 masters and Ph.D. candidates. After more than a century of operation, Jiao Tong University has inherited the old tradition of "high starting points, solid foundation, strict requirements and extensive practice." Students from SJTU have won top prizes in various competitions, including ACM International Collegiate Programming Contest, International Mathematical Contest in Modeling and Electronics Design Contests. Famous alumni include Jiang Zemin, Lu Dingyi, Ding Guangen, Wang Daohan, Qian Xuesen, Wu Wenjun, Zou Taofen, Mao Yisheng, Cai Er, Huang Yanpei, Shao Lizi, Wang An and many more. More than 200 of the academics of the Chinese Academy of Sciences and Chinese Academy of Engineering are alumni of Jiao Tong University.

  \englishkeywords{\large SJTU, master thesis, XeTeX/LaTeX template}
\end{englishabstract}


% 目录
\tableofcontents
% 插图索引
\listoffigures
\addcontentsline{toc}{chapter}{\listfigurename} %将图索引加入全文目录
% 表格索引
\listoftables
\addcontentsline{toc}{chapter}{\listtablename}  %将表格索引加入全文目录

% 主要符号、缩略词对照表
%%==================================================
%% symbol.tex for SJTU Master Thesis
%% based on CASthesis
%% modified by wei.jianwen@gmail.com
%% version: 0.3a
%% Encoding: UTF-8
%% last update: Dec 5th, 2010
%%==================================================

\chapter{主要符号对照表}
\label{chap:symb}
\begin{tabular}{ll}

 \hspace{2em}$\epsilon$       & \hspace{5em}介电常数 \\
 \hspace{2em}$\mu$ \qquad     & \hspace{5em}磁导率 \\
  \hspace{2em}$\epsilon$       & \hspace{5em}介电常数 \\
 \hspace{2em}$\mu$ \qquad     & \hspace{5em}磁导率 \\
 \hspace{2em}$\epsilon$       & \hspace{5em}介电常数 \\
 \hspace{2em}$\mu$ \qquad     & \hspace{5em}磁导率 \\
 \hspace{2em}$\epsilon$       & \hspace{5em}介电常数 \\
 \hspace{2em}$\mu$ \qquad     & \hspace{5em}磁导率 \\


\end{tabular}


%%%%%%%%%%%%%%%%%%%%%%%%%%%%%% 
%% 正文
%%%%%%%%%%%%%%%%%%%%%%%%%%%%%% 
\mainmatter


%% 各章正文内容

%\bibliographystyle{sjtu2} %[此处用于每章都生产参考文献]
\chapter{绪论}
\label{chap:intro}

\section{研究背景}
\label{sec:background}

在计算机科学理论的发展过程中,许多不同的计算模型被相继提出。最初的计算模型所定义的都是串行计算。例如图灵机(Turing Machine)\cite{Turing1936}, $\lambda$-演算($\lambda$-Calculus)\cite{Church1985}, 递归函数(Recursive Function)\cite{Rogers1967}等。随着并行化计算的发展,理论计算机科学家提出了并行化的计算模型并进行了深入的研究。其中比较有代表性的是由R.Milner提出的Communication Concurrency System (CCS)\cite{Milner1989}。该模型利用进程演算的方法对可以实现并发和交互的进程模型进行了刻画。

在这些计算模型的研究中,涉及到一个计算机科学中十分重要的领域:形式化验证(Formal Verification)。而这些计算模型,大多数都可以描述无限状态系统。对无限状态系统的形式化验证(Verification on Infinite State Systems),是现今理论计算机科学中的一个热门的研究方向,它包含了一系列具有重要意义的研究课题。这些问题往往可以与可计算理论(Computability), 计算复杂性理论(Computational Complexity),算法(Algorithm)等领域中的一些经典问题相联系起来,从而得出许多振奋人心的可计算性或复杂性结论。

为了对这些模型进行验证,我们需要选择合适的等价关系(Equivalence Relation)。人们最初研究的等价关系是语言等价(Language Equivalence),然而即使对于上下文无关语言(Context Free Language),其语言等价也是不可判定的\cite{Hopcroft1979}。而不可判定的等价关系从验证角度来说是用处不大的。于是许多基于观测理论(Observation Theory) 中互模拟(Bisimulation) 概念的等价关系被提出,最早的是由 Park 提出的强互模拟(Strong Bisimulation)\cite{Park1981}。随后,为了区别系统中内部动作和外部动作,Milner引入了$\tau$动作表示系统内部的转换,并且定义了弱互模拟(Weak Bisimulation)\cite{Milner1989}。为了更加精细地区分$\tau$动作对系统状态的影响,van Glabbeek和 Weijland提出了Branching 互模拟(Branching Bisimilarity)\cite{Glabbeek1996}。这些基于互模拟的等价关系,是研究验证问题的理论基础。

在研究无限状态系统时,系统的模型可以使用进程(Process)进行表示。对于各种不同种类的无限状态系统,都可以利用一个统一的进程代数模型进行表示,即进程重写系统(Process Rewrite System, 简称PRS)\cite{Mayr2000}。PRS是一个具有一般性的进程模型,它提供不同模型的关于强互模拟的表达能力的层次结构。许多常见的进程代数模型都可以在这个层次结构中找到。

而对于一个具体的模型所描述的系统,确定需要研究的等价关系后,我们所关心的问题主要分3类,分别是关于该等价关系的等价性判定(Equivalence Checking),与某个给定的有限状态系统的等价性判定(Finiteness)和是否存在一个有限状态系统与该系统等价(Regularity)。这3类问题在本质上有着一定的联系,但是在解决的方法和难度上却有着区别。本文将重点对第三类问题,即有限性问题(Regularity Problem)进行讨论。

从直观的角度解释,Regularity问题是给定一个可描述无限状态系统的模型是否可表示为等价的有限系统的判定问题。在某种意义上,它也是等价判定问题和模型检测问题的一个重要条件。另一个振奋人心的事实是,Regularity性质的成立与否,决定了系统所能到达的不同状态是否是有限的,该条件如果成立,相应的判定问题通常都存在快速解决的算法。

\section{国内外研究现状}
\label{sec:state-of-the-art}

无限状态系统的验证近年来一直是理论计算机科学中的一个十分活跃的研究领域。在这方面最早的可判定性结论是由Baeten, Bergstra和 Klop证明的上下文无关语法关于强互模拟等价的可判定性\cite{Baeten1993}。 这一结论引发了对于各种不同无限状态系统的等价性判定的研究,许多问题的可判定性和复杂性结论被提出并得到了证明。我们可以找到许多相关的调查研究\cite{Burkart2001, Kucera2006, Moller2004, Srba2002}。

这些研究所涉及到的大多数模型都可以由进程重写系统所产生\cite{Mayr2000},这些研究都极大的提升了我们对于一些经典无限状态系统模型的认识。这个框架下包括了许多我们熟悉的模型,如基本进程代数(Basic Process Algebra, 简称BPA)\cite{Bergstra1985},基本并行进程(Basic Parallel Process, 简称BPP)\cite{Christensen1993},进程代数(Process Algebra,简称PA)\cite{Baeten1990},下推自动机(Pushdown Automaton,简称PDA)\cite{Hopcroft1979}以及Petri网(Petri Net,简称PN)\cite{Peterson1977}。这些模型关于互模拟等价的表达能力在进程重写系统中产生了一个严格包含关系的层次结构,这种结构使得我们的研究更加具有效率。例如一个模型的复杂性上界结论可以直接隐含其子模型上的结论,而如果对于某个模型中一个问题,我们能在其某个子模型上证明它的复杂性下界,那这个结论在该模型上显然也成立。以上即为本文中的研究所涉及到的模型。

在这个无限状态系统验证的领域中,人们最初的兴趣总是集中在对Equivalence Checking问题的研究。经过20年的发展,关于强互模拟关系的判定研究已经比较成熟。最早的BPA上的判定性算法是通过对进程的分解技术实现的\cite{Baeten1993,Christensen1992},经过对该技术的改进和对模型的进一步限定,在Normed BPA和Normed BPP上,强互模拟都有了多项式时间的判定性算法\cite{Hirshfeld,Hirshfelda},在这里,Normed是对于进程模型的一个限定条件,它规定了模型必须可以到达空进程。同时,对于一般的BPA和BPP,2-EXPTIME和PSPACE完全的时间复杂性也分别被证明\cite{Burkart1995,Jancar2012,Jancar2003}。对于PDA,强互模拟是可判定的\cite{Senizergues1998,Stirling1998},同时对Normed PA强互模拟的判定也被证明是在2-NEXPTIME的时间复杂度内可判定的\cite{Hirshfeldb}。然而,PN的强互模拟等价性即使在Normed条件的限定下也是不可判定的\cite{Jancar1995}。 

但是在实际的系统中,例如在程序分析和数据库系统中,很多系统的转换只能用内部动作来描述,所以更具有实际意义的通常是区分系统内部动作的弱互模拟和Branching互模拟。许多模型关于带有内部动作的互模拟等价关系判定问题都被证明是不可判定的\cite{Jancar2008}。然而,对于Branching互模拟的判定问题近两年得到了很大的突破。Normed BPP和Normed BPA上关于Branching互模拟等价的可判定性分别被证明\cite{CzerwiAski2011,Fu2013}。这两个结论是十分令人兴奋的,它们为Branching互模拟等价的相关研究开辟了新的道路。

Finiteness的判定是一个和Regularity相似但是直观上更加容易的问题。因为该问题中有限状态系统是给定的,我们需要做的仅仅是判定它和给定系统是否等价。关于各种互模拟关系的Finitness问题通常都有多项式时间复杂度的快速解法。例如BPA和Normed BPP中关于弱互模拟和Branching互模拟的finiteness都有多项式时间算法\cite{Kucera2002,Fu2009}。即使没有多项式时间的算法,关于PDA,PA和PN关于强互模拟分别是PSPACE\cite{Kucera2002a},co-NEXPTIME\cite{Goller2011}和可判定的\cite{Jancar1995a}。显然,通常这些问题都比对应的Equivalence Checking问题有着更好的计算复杂性和可判定性。而Regularity问题在某种意义上提供了联系这两个问题的一个桥梁:如果某个进程满足Regularity性质,那么我们就可以用更快速的Finiteness判定的算法来进行Equivalence Checking的工作。

在现阶段,关于强互模拟关系的Regularity问题通常在可判定性上有了不错的结果。在BPA上关于强互模拟关系的Regularity问题被证明是2-EXPTIME的\cite{Burkart1995,Burkart1996},而对于BPP则是PSPACE完全的\cite{Kot2005a}。在PDA上,现阶段只证明了Normed条件下的可判定性,而且有一个多项式时间的算法\cite{Esparza2000}。关于PA,也仅在Normed条件下有一个多项式时间的算法\cite{Kucera1996}。而对于PN,我们有Regularity的可判定性,并且在引入内部动作后该问题变为不可判定的\cite{Jancar1995a}。

在互模拟等价关系引入内部动作之后,我们现阶段已知的Regularity问题的可判定结论就十分有限了。现在唯一具有实际意义的结论就是由Fu在2013年证明的,关于Branching互模拟,在Normed BPA上Regularity问题的可判定性\cite{Fu2013}。

为了简化模型,有的时候研究的模型可以加上Totally Normed条件。在该限定下关于弱互模拟和Branching互模拟的问题通常更好的可判定行结论或者算法。H\"{u}ttel最早在\cite{Huttel1992}中引入了该限定,并证明了Totally Normed BPP的Branching互模拟的可判定性。Chen也在这一限定条件下证明了Totally Normed BPA和Totally Normed BPP关于弱互模拟的可判定性\cite{Chen2008a,Chen2008}。

\section{主要工作}
\label{sec:contribution}

本文主要针对关于引入内部动作的互模拟等级关系的Regularity问题进行了讨论。主要贡献可以分为一下几个方面:

\begin{enumerate}
	\item 归纳总结了PRS上Regularity问题的现有结论与技术,给出了一些解决Regularity问题所用到的技术和引理。
	\item 对Totally Normed PA关于弱互模拟和Branching互模拟关系的Regularity问题给出了一个多项式时间算法。证明了该算法的正确性,并做了复杂度分析。该算法的时间复杂度是$\mathscr{O}(n^3+mn)$的,其中$m$和$n$都是和输入模型相关的参数。该算法对PA的子模型BPA和BPP也成立。
	\item 对Normed BPP关于Branching互模拟关系以及Totally Normed PN关于弱互模拟和Branching互模拟关系的Regularity问题进行了讨论,并为今后的进一步工作提供了一定的思路。
	\item 针对Regularity问题的下界研究提出了一些现阶段似乎可以的问题,并且直接归约出了一些有用的推论。
\end{enumerate}

本文的工作讨论的问题是属于无限状态系统验证领域中的一部分工作,和该领域中很多经典的问题都有交叉。在得到了关于Totally Normed PA上不错结论的同时,也提出了对后续问题的解决十分具有启发意义的研究思路。

\section{章节安排}
\label{sec:section}

在本文的第\ref{chap:pre}章中,将具体介绍本文讨论的问题所涉及到的进程重写系统中的各种模型以及所用到的各种互模拟等价关系;第\ref{chap:relat}章中,将会介绍Regularity问题一些已有的结论和解决Regularity问题所需要用到的技术和一些引理,以及直接得到的推论;第\ref{chap:tnpa-equiv}章中,将会给出Totally Normed PA关于弱互模拟和Branching互模拟的一个等价条件;第\ref{chap:tnpa-alg}章中,将会给出该问题的多项式时间算法,并做计算复杂性分析;第\ref{chap:fut}章中,将对后续准备解决的问题决进行一些讨论;最后,将对全文的工作进行总结。

%%==================================================
\chapter{背景知识}
\label{chap:background}

\section{进程重写系统 PRS}
\label{sec-prs}

\subsection{基本定义}
进程重写系统(PRS)是一个可以用来刻画进程模型语义的一般系统,这一部分我们会给出关于进程重写系统的一些基本定义。

在此之前,我们可以分析一个进程代数中的例子\cite{Milner1989},以下定义了一个计数器(Counter Machine)的\emph{需求(Specification)}。
\begin{eqnarray*}
C_{0} &=& zero.C_{0}+inc.C_{1}, \\
C_{i+1} &=& dec.C_{i}+inc.C_{i+2},\ \mathrm{where}\ i\ge0.
\end{eqnarray*}
下面是Busi, Gabbrielli 和 Zavattaro 给出的\emph{实现(Implementation)}\cite{Busi2003}:
\begin{eqnarray*}
Counter &=& zero.Counter+inc.(d)(O\,|\,d.Counter), \\
O &=& dec.\overline{d}+inc.(e)(E\,|\,e.O), \\
E &=& dec.\overline{e}+inc.(d)(O\,|\,d.E).
\end{eqnarray*}
用BPA来\emph{编码(Programming)}就是:
\[
Z \stackrel{inc}{\longrightarrow} XZ, \ \ \
Z \stackrel{zero}{\longrightarrow} Z, \ \ \
X \stackrel{inc}{\longrightarrow} XX, \ \ \
X \stackrel{dec}{\longrightarrow} \epsilon.
\]
通过这个例子,我们可以直观的看出PRS中的BPA模型可以编码一个计数器。当然我们也可以通过更加复杂的编码来实现更加复杂工作的验证。我们下面给出PRS语法的定义和语义的规则。

\begin{defn}[进程项 Process Term]\label{def-process term}
令$Act=\{a,b,\ldots\}$是一个\emph{原子动作(atomic actions)}的集合;$Const=\{\epsilon\}\cup\{X,Y,Z,\ldots\}$是一个 \emph{进程常量(process constants)}的集合。
$S=\{\alpha_1,\alpha_2,\ldots\}$被称为\emph{进程项(process terms)}的集合,它被用来刻画系统的状态,可以由一下的BNF产生:
$$\alpha\Coloneqq \epsilon\mid X\mid \alpha_1.\alpha_2\mid \alpha_1\para \alpha_2$$
其中
\begin{itemize}
\item $\epsilon$被称为\emph{空进程(empty process)};
\item $\alpha_1.\alpha_2$是一个\emph{串行(sequential)}进程;
\item $\alpha_1\para \alpha_2$是一个\emph{并行(parallel)}进程。
\end{itemize}
我们这里用小写希腊字母$\alpha,\beta,\gamma,\ldots$来表示进程项。
\end{defn}

有了进程项的定义,对于一个进程演算系统,就定义它的\emph{操作语义(Operational Semantics)}。这里,我们使用\emph{标号迁移系统(Labeled Transition System 简称LTS)}来定义PRS中的模型所遵循的语义规则。

\begin{defn}[标号迁移系统 LTS]\label{def-lts}
一个\emph{标号迁移系统(LTS)}是一个五元组$(S,Act,\longrightarrow,\alpha_0,F)$,其中
\begin{itemize}
    \item $S$是一个\emph{状态(states)}的有限集合,
    \item $Act$是一个\emph{标号(labels)}的有限集合,
    \item $\longrightarrow\;\subseteq S\times Act\times S$是一个\emph{转换关系(transition relation)},
    \item $\alpha_0\in S$是一个给定的\emph{初始状态(start state)},
    \item $F\subseteq S$是一个\emph{终结状态(final states)}的有限集合,这意味着对于任何$\alpha\in F$不存在$a\in Act$和$\beta\in S$使得$\alpha\stackrel{a}{\longrightarrow}\beta$。
\end{itemize}
我们通常将$(\alpha,a,\beta)\in \longrightarrow$记做$\alpha\stackrel{a}{\longrightarrow}\beta$。
\end{defn}

接下来就可以利用\emph{语义推导规则(inference rules)}来得到PRS模型的操作语义。LTS所定义的语义转换关系是由形如$\alpha\act{a}\beta$的规则的所构成的有限集合$\Delta$所生成的。对于任意$a\in Act$,语义迁移关系$\act{a}$是从以下的语义推导规则构造的最小的转换关系:
$$\begin{tabular}{cc}
$\inference{\alpha\stackrel{a}{\longrightarrow}\beta\in\Delta}{\alpha\stackrel{a}{\longrightarrow}\beta}$&
$\inference{\alpha\stackrel{a}{\longrightarrow}\alpha'}{\alpha.\beta\stackrel{a}{\longrightarrow}\alpha'.\beta}$ \\
\\
$\inference{\alpha\stackrel{a}{\longrightarrow}\alpha'}{\alpha\para \beta\stackrel{a}{\longrightarrow}\alpha'\para \beta}$ & $\inference{\beta\stackrel{a}{\longrightarrow}\beta'}{\alpha\para \beta\stackrel{a}{\longrightarrow}\alpha\para \beta'}$
\end{tabular}$$

\section{互模拟等价关系}
\label{sec-bis}


%%==================================================

\chapter{相关结论和技术}
\label{chap:relat}

为了能更好的研究PRS上的Regularity问题,本章中我们首先将引入一个无限状态系统验证问题中常常用到的重要限定条件,Normed进程,同时我们会对该限定做进一步的规定,并给出Totally Normed进程。之后我们将介绍PRS上Regularity问题现阶段我们已知的结论,并进行归纳和总结,并得到一些推论。最后将介绍一些PRS上证明Regularity问题所用到的一些技术和引理。


\section{Normed条件}
\label{sec:norm}
 
在无限状态系统验证问题中,我们常常要对所研究的模型做一些合理的限定,以便使该模型满足一些我们所需要的性质,从而可以得到更好的可判定行或者算法复杂性的结论。从研究语言理论的角度出发,我们经常要求模型是\emph{Normed}。从直观上来说,一个Normed进程可以通过有限步转换到达一个$\epsilon$进程。如果做进一步限定,Totally Normed进程则不仅要满足Normed性质,而且到达$\epsilon$进程至少要经过一个可见动作。(Totally) Normed进程的定义为:

\begin{defn}[(Totally) Normed 进程 (Totally)Normed Process]\label{def:norm}
一个进程项$\alpha$的Norm,记做$\|\alpha\|$,表示从$\alpha$能达到$\epsilon$的转换序列的最短长度,如果考虑弱互模拟,则不考虑$\tau$动作。

我们称一个进程项$\alpha$是Normed,如果满足$0\leq\|\alpha\|<\infty$,称它是Totally Normed,如果满足$0<\|\alpha\|<\infty$。
一个Normed PRS进程$\Delta$中所有的进程变量均是Normed。
类似的,一个Totally Normed PRS进程$\Delta$中所有的进程变量均是Totally Normed。
\end{defn}

Normed条件规定了一个进程只能做有限个减少Norm的动作,满足该条件的进程往往会满足一些美妙的分解性质。而Totally Normed条件则进一步限定了进程做$\tau$动作的限制,没有进程项能只做$\tau$动作就变空。Totally Normed限定也是十分合理的,在现实系统中通常也不需要定义,或者通过一些技巧来规避那种``悄无声息''就消失的进程。这些限定在研究考虑内部动作的互模拟关系的验证问题时,发挥了很大的作用。

为了表达的简洁性,我们在相关模型前加上n和tn分别作为Normed和Totally Normed的简称。例如Normed BPA简称为nBPA,而Totally Normed PA简称为tnPA。

\section{PRS上Regularity问题现有结论总结}
\label{sec:prs-reg-result}

本节将根据不同的PRS子模型,分别总结其现有的关于强互模拟,弱互模拟和Branching互模拟的Regularity问题的结论,并根据问题的相似性,直接给出一些推论。根据定义\ref{def:prob}中给出的记号,我们将三类问题分别及做$\seq_{REG},\,\weq_{REG}$和$\beq_{REG}$。

表\ref{tab:bpa-reg}为BPA上的结论总结,表中如果出现两行则分别表示该问题已知的复杂度上界和下界。``?''表示该问题还未被解决。
\begin{table}[htbp]
\begin{center}
\begin{tabular}{|c|c|c|}
\hline
		&\BPA 			& \nBPA \\ 
\hline
\hline
$\sim_{REG}$ 	&\begin{tabular}{c} 
		Decidable~\cite{Burkart1995,Burkart1996} \\
		PSPACE-hard~\cite{Srba2002b} 
		\end{tabular} 		
					& NL-complete~\cite{Srba2002b}\cite{Kucera1996} \\ 
\hline
$\beq_{REG}$	&\begin{tabular}{c} ? 
		\\
		EXPTIME-hard~\cite{Mayr2003} 
		\end{tabular} 
					& Decidable~\cite{Fu2013} \\
\hline
$\weq_{REG}$ 	&\begin{tabular}{c} ? 
		\\
		EXPTIME-hard~\cite{Mayr2003} 
		\end{tabular} 
					&\begin{tabular}{c} ? 
					\\ 
					NP-hard~\cite{Srba2003,StriAbra1998} 
					\end{tabular} \\ 
\hline
\end{tabular}
\caption{\textsf{BPA} Regularity 问题现有结论}
\label{tab:bpa-reg}
\end{center}
\end{table}

其中由Fu证明的$nBPA$上$\beq_{REG}$问题的可判定行结论是现在唯一已知的考虑系统内部动作的互模拟等价关系的Regularity问题结论。

表\ref{tab:bpp-reg}总结了BPP上的已知结论
\begin{table}[htbp]
\begin{center}
\begin{tabular}{|c|c|c|} 
\hline
		&BPP			&nBPP \\ 
\hline
\hline
$\seq$		&
		PSPACE-complete~\cite{Kot2005a,Jancar1996,Srba2002a} 
					&\begin{tabular}{c} 
					NL~\cite{Kucera1996} \\ 
					NL-hard~\cite{Srba2002a} 
					\end{tabular} \\ 
\hline
$\beq$		& ? 			&? \\ 
\hline
$\weq$		&\begin{tabular}{c} 
		? \\ 
		PSPACE-hard~\cite{Srba2003} 
		\end{tabular}		&\begin{tabular}{c} 
					? \\ 
					PSPACE-hard~\cite{Srba2003} 
					\end{tabular} \\ 
\hline
\end{tabular}
\end{center}
\caption{\textsf{BPP} Regularity 问题现有结论}
\label{tab:bpp-reg}
\end{table}

在BPP上,我们已经有了$\seq_{REG}$的PSPACE完全性结论。这里利用到了一种在BPP中常用的DD-Function的技术,来对BPP的等价类进行划分,但是,对$\weq$和$\beq$,DD-Function并不适用,所以我们还需要寻找其他技术来证明其可判定行。

表\ref{tab:pda-reg}为PDA上的现有结论

\begin{table}[htbp]
\begin{center}
\begin{tabular}{|c|c|c|} 
\hline
		&PDA		&nPDA \\ 
\hline
\hline
$\seq$ 		&\begin{tabular}{c} 
		? \\ 
		EXPTIME-hard~\cite{Kucera2002a,Srba2002b} 
		\end{tabular} 	& \begin{tabular}{c} 
				P~\cite{Esparza2000} \\ 
				NL-hard~\cite{Srba2002b} 
				\end{tabular} \\ 
\hline
$\beq$		& ? 		& ? \\ 
\hline
$\weq$ 		&\begin{tabular}{c} 
		? \\ 
		EXPTIME-hard~\cite{Kucera2002a,Srba2002b}
		\end{tabular} 	& \begin{tabular}{c} 
				? \\ 
				EXPTIME-hard~\cite{Kucera2002a,Srba2002b} \end{tabular} \\ 
\hline
\end{tabular}
\end{center}
\caption{\textsf{PDA} Regularity 问题现有结论}
\label{tab:pda-reg}
\end{table}

在PDA上,我们似乎还没有找到一些好的方法来处理Regularity问题,即使对$\seq_{REG}$。

表\ref{tab:pn-reg}总结了PN上的现有结论

\begin{table}[htbp]
\begin{center}
\begin{tabular}{|c|c|c|} 
\hline
		&PN			&nPN \\ 
\hline
\hline
$\seq$ 		&\begin{tabular}{c} 
		Decidable~\cite{Jancar1996} \\ 
		PSPACE-hard~\cite{Srba2002a} 
		\end{tabular}		&\begin{tabular}{c} 
					EXPSAPCE~\cite{Rackoff1978} \\ 
					EXPSPACE-hard~\cite{Cardoza1976} 
					\end{tabular} \\ 
\hline
$\beq$ 		&Undecidable*	&? \\ 
\hline
$\weq$ 		&\begin{tabular}{c} 
		Undecidable~\cite{Jancar1996} \\ 
		EXPSPACE-hard~\cite{Cardoza1976} 
		\end{tabular} 		&\begin{tabular}{c} 
					? \\ 
					EXPSPACE-hard~\cite{Cardoza1976} 
					\end{tabular} \\ 
\hline
\end{tabular}
\end{center}
\caption{\textsf{PN} Regularity 问题现有结论}
\label{tab:pn-reg}
\end{table}

这里我们注意*所标的结论,即PN上$\beq_{REG}$的不可判定性并未直接在引用论文中出现,而是本文作者经过考证和与人讨论,将$\weq_{REG}$不可判定性的证明技术,应用到$\beq_{REG}$上,得出的一个引文\cite{Jancar1996}的推论。

\begin{cor}[PN $\beq_{REG}$ 的不可判定性]\label{cor:pn-beq-und}
PN的$\beq_{REG}$问题是不可判定(Undecidable)的。
\end{cor}

最后表\ref{tab:pa-reg}将总结PA上的相关结论,因为我们将要研究的问题是tnPA上的,所以这里同时还总结了tnPA上的结论

\begin{table}[htbp]
\begin{center}
\begin{tabular}{|c|c|c|c|}
\hline
	&PA	&nPA	&tnPA\\
\hline
\hline
\multirow{2}{*}{$\seq_{reg}$} &
 ? &NL~\cite{Kucera1996} &NL~\cite{Kucera1996} \\
& PSPACE-hard~\cite{Srba2002a} & NL-hard~\cite{Srba2002a} & NL-hard~\cite{Srba2002a}\\
\hline
\multirow{2}{*}{$\beq_{reg}$} &
 ? & ? &P* \\
& EXIPTIME-hard~\cite{Mayr2003} & PSPACE-hard~\cite{Srba2003} & NL-hard~\cite{Srba2002a}\\
\hline
\multirow{2}{*}{$\weq_{reg}$} &
 ? & ? &P* \\
& EXPTIME-hard~\cite{Mayr2003} & PSPACE-hard~\cite{Srba2003} & NL-hard~\cite{Srba2002a}\\
\hline
\end{tabular}
\caption{\textsf{PA} Regularity 问题现有结论}
\label{tab:pa-reg}
\end{center}
\end{table}

其中*所标的两个多项式时间判定算法为本文的主要结论,该结论同样适用于PA的子模型tnBPA和tnBPP。
\section{相关技术路线和引理}
\label{sec:lemma}

\chapter{Totally Normed PA Regularity的充要条件}
\label{chap:tnpa-equiv}

在这一章中,我们将具体讨论和证明tnPA的$\weq_{REG}$和$\beq_{REG}$问题的充分必要条件,定理\ref{thm:tnpa-equiv}。这里将\cite{Kucera1996}中用于判定nPA的$\seq_{REG}$的一些技术应用到我们讨论的问题中,起到了很好的效果。这里证明的充要条件将会在第\ref{chap:tnpa-alg}章中的多项式算法中被判定。

这里需要说明的是,由于这些条件都需要是可判定的,在判定的过程中我们需要一个计算复杂性上可行的算法。这就意味着我们所给出条件的判定过程需要有一个确定的,有限的,可以有效计算的上界。否则判定算法上会发生状态数量的爆炸,整个过程就丧失了计算上的可行性。这个上界会在接下来的分析中一步步变得清晰,我们要遵守的一个原则是这个上界只能和输入进程的语法规则和初始状态的规模相关。

\section{PA模型定义}
\label{sec:pa-def}

为了问题研究的严谨性,我们这里在PRS的一般性定义\ref{def:prs}的基础上,给出PA的严格定义。

\begin{defn}[PA 定义]\label{def:pa}
一个PA系统是一个形如$X\act{a}\alpha$的有限转换规则的集合$\Delta$,一个PA表达式的语法定义为:
$$\alpha,\beta\Coloneqq X\mid \alpha.\beta\mid\alpha\parallel\beta\mid\alpha\llfloor\beta$$
其中用大写英文字母表示进程变量,希腊字母表示进程项(状态)。
相应的操作语义可以定义为一下规则
\[\begin{array}{ccc}
$\inference{\alpha\act{a}\beta\in\Delta}{\alpha\act{a}\beta}$ &
$\inference{\alpha\act{a}\alpha^{'}}{\alpha.\beta\act{a}\alpha^{'}.\beta}$ &
$\inference{\alpha\act{a}\alpha^{'}}{\alpha\| \beta\act{a}\alpha^{'}\|\beta}$ \\
 \\
$\inference{\beta\act{a}\beta^{'}}{\alpha\| \beta\act{a}\alpha \| \beta^{'}}$ &
$\inference{\alpha\act{a}\alpha^{'}}{\alpha\llfloor \beta\act{a}\alpha^{'}\|\beta}$ &
\end{array}\]
一个PA进程可以具体表示为$(\alpha,\Delta)$。
\end{defn}

我们用函数$Var(\_)$表示$\_$中所包含的进程变量。

我们要求一个PA中定义的所有进程变量$X_i\in Var(\Delta)$存在一个可达的状态$\alpha$满足$\X_i\in Var(\alpha)$。

另外,我们用$Length(\alpha)$表示$\alpha$中出现的进程变量的个数。

同时,我们限制在章里我们所关心的PA进程都是Totally Normed的。同时如果不特殊说明,我们证明中所用到的等价关系都是弱互模拟关系$\weq$。这些证明对于$\beq$都是成立的。

\section{进程状态的增长}
\label{sec:grow-prop}

给定一个tnPA的进程$(\alpha,\Delta)$,我们希望能找到导致它产生无限中不同状态的进程变量的集合。这些变量导致了进程状态所能产生的增长,是导致进程非Regularity的原因。

我们首先针对一个tnPA进程,定义两个十分有用的函数。

\begin{defn}[Fire 函数]\label{def:fire}
给定一个tnPA进程$(\alpha,\Delta)$
\begin{eqnarray*}
Fire(\alpha) &=& \left\{ \begin{array}{ll}
\emptyset & \mbox{if }\alpha=\epsilon\\
\{X\} & \mbox{if }\alpha=X\\
Fire(\beta_1) & \mbox{if }\alpha=\beta_1.\beta_2\mbox{ or }\alpha=\beta_1\llfloor\beta_2\\
Fire(\beta_1)\cup Fire(\beta_2) & \mbox{if }\alpha=\beta_1\|\beta_2
\end{array}\right.
\end{eqnarray*}
\end{defn}

$Fire()$函数会返回那些当前状态下潜在可以被激发(Active)的进程变量的集合。

\begin{defn}[Tail 函数]\label{def:tail}
给定一个tnPA进程$(\alpha,\Delta)$
\begin{eqnarray*}
Tail(\alpha) &=& \left\{ \begin{array}{ll}
\{X\} & \mbox{if }\alpha=X\\
\emptyset & \mbox{if }\alpha=\epsilon \mbox{ or } \alpha=\beta_1\|\beta_2 \mbox{ where } \beta_1\neq\epsilon\neq\beta_2\\
Tail(\beta_2)-Var(\beta_1) &\mbox{if }\alpha=\beta_1.\beta_2 \mbox{ or }\alpha=\beta_1\llfloor\beta_2\mbox{ where }\beta_1\neq\epsilon\neq\beta_2
\end{array}\right.
\end{eqnarray*}
\end{defn}

$Tail()$函数返回所有在当前状态下被阻塞(Block),即必须先将别的变量转换成$\epsilon$,最后才能激发的进程变量的集合。

通过观察这两个函数的性质,我们能得到以下性质,引理\ref{lemma:tail}:

\begin{lem}\label{lemma:tail}
假设我们有$X\in Var(\alpha)$满足$X\notin Tail(\alpha)$,那么必然存在$\alpha'$,使得$\alpha\act{}^{*}\alpha'$,且$X\in Fire(\alpha')$且$Length(\alpha')\geq 2$。
\end{lem}

\begin{proof}
用反证法,如果$X\in Fire(\alpha')$且$Length(\alpha')=1$。那么$X$可以会在别的进程变量都消失后再被激发,就有$X\in Tail(\alpha)$,矛盾。
\end{proof}

同时,引理\ref{lemma:fire}也是十分自然的:

\begin{lem}\label{lemma:fire}
给定tnPA进程$(\alpha,\Delta)$。那么对于所有$X\in Var(\alpha)$,存在$(\beta,\Delta)$,满足$\alpha\act{}^{*}\beta$,且$X\in Fire(\beta)$。
\end{lem}

\begin{proof}
由$Fire$函数的定义直接显然成立。
\end{proof}

在我们考虑解决tnPA进程的Regularity问题时,希望找到一个充分必要条件。由引理\ref{lemma:infi-path-2}中介绍的必要条件,我们首先希望能找到那种有潜在生成无限种不同语言(Lexically)上不同的进程表达式能力的进程变量。然后我们再去证明这些表达式间的互不互模拟性,从而证明其充分性。

根据这种思路,我们给出关于能使进程\emph{增长(Growing)}的进程变量的定义\ref{def:grow-var}:

\begin{defn}[增长变量 Growing Variable]\label{def:grow-var}
给定一个tnPA进程的规则集合$\Delta$,对于一个变量$X\in Var(\Delta)$,如果存在$\alpha$,满足$X\act{}^{*}\alpha$,$X\in Fire(\alpha)$且$\Length(\alpha)\geq 2$。那么我们称$X$是\emph{增长的(Growing)}。
\end{defn}

\section{转换关系树}
\label{sec:trans-tree}

接下来,我们将会证明在tnPA中,Growing的进程变量就是导致进程状态无限性的原因。再回顾一下引理\ref{lemma:infi-path}中的提供的思路,当我们希望验证状态的无限性时,我们只需要考虑那种无限长的形如$\alpha\act{a_0}\alpha_1\act{a_1}\alpha_2\act{a_2}\ldots$,且满足$\alpha_i\not\weq \alpha_j$ for $i\neq j$的动作路径\mathcal{P}。为了考察路径中那些进程表达试的长度,我们将使用一个树状的结构对其进行表示,我们称之为\emph{转换关系树(Transition Tree)},记做$T_{\mathcal{P}}$,在定义\ref{def:trans-tree}中利用归纳法,我们给出其严格的定义。利用$T_{\mathcal{P}}$,我们可以准确刻画路径$\mathcal{P}$中连续出现的变量之间的关系。这些信息对我们接下来的证明是有很大用处的。

\begin{defn}[转换关系树]\label{def:trans-tree}

给定一个tnPA进程$(\alpha,\Delta)$和一个形如$\alpha\act{a_0}\alpha_1\act{a_1}\alpha_2\act{a_2}\ldots$.的状态转换路径$\mathcal{P}$
一个\emph{转换关系树(Transition Tree)},$T_{\mathcal{P}}=(V,E)$定义如下:
\begin{enumerate}
\item $V$ 是结点的集合,它们由$Var(\Delta)\cup \{R\}$进行标号,其中 $R$表示树根的特殊标号。
\item $E$ 是树中边的集合
\item 路径$\mathcal{P}$中每一个状态,例如$\alpha_j$,被树里的第$j+1$层记录,记做$Level_{j+1}$。这层中的节点被$\alpha_j$中的变量所标记,如果重复出现,则重复标记。
\end{enumerate}

$T_{\mathcal{P}}$的拓扑结构(Topological Structure)可以利用树的层数$i$进行归纳定义。
\begin{enumerate}
\item 当$i=0$:为树根,标记为$R$。
\item 当$i=1$:这一层对应了进程的初始状态$\alpha_0$。
\item 当$i>1$: 我们假设$Level_i$ 上所有节点都已经被定义,接着我们试图定义$Level_{i+1}$上的所有节点。我们现在定义$Level_i$上所有节点的子节点的规则:

根据定义\ref{def:pa}中PA的操作语义,我们知道对于一步转换$\alpha_{i-1}\act{a_i}\alpha_{i}$,一定存在一个变量,假如是$A\in Var(\alpha_{i-1})$和一条转换规则$A\act{a_i}\gamma_i\in\Delta$被激发。我们称这个变量$A$为$\alpha_{i-1}$中的\emph{激发变量(Active Variable)},对应的树中的节点为$Level_i$中的\emph{激发节点(Active Node)}。

我们根据此将$Level_i$中所有节点$N\in V$分成两类分别定义其子节点的标号规则:
\begin{enumerate}
\item 如果$N$不是一个激发节点。那么$N$只有一个儿子节点,标号和$N$相同。而且,一条在它们之间连一条边来表示他们之间的标号继承关系。
\item 如果$N$是一个激发节点。我们假设它对应的进程变量为$A$,而相应的转换规则为$A\act{a_i}\gamma_i$。令$n=Length(\gamma_i)$。如果$n>0$,那么节点$N$有$n$个子节点,否则它是一个叶子节点。它的第$k$个子节点的标号就是$\gamma_i$中从左到右的第$k$个变量。
\end{enumerate}
\end{enumerate}
\end{defn}

观察这个定义,我们可以看出进程转换中标号变量的继承关系的信息可以在树中表示出,而且保留了一部分的进程的结构信息。下面的引理\ref{lemma:children}告诉了我们这样的子节点的排列给我们带来的一些信息。

\begin{lem}\label{lemma:children}
给定一个转换关系树$T_{\mathcal{P}}$,$N$是$\Level_{i+1}$层中的一个标号为$A$的激发节点,其对应的规则是$A\act{a_i}\gamma_i$。它的儿子节点从左到右分别为$B_1,B_2,\ldots,B_n$。假设 $Tail(\gamma_i)=\{B\}$,那么$N$的最后一个子节点的标号就是被阻塞的进程变量$B$。
\end{lem}

\begin{proof}
很明显,这个性质可以由定义\ref{def:tail}和定义\ref{def:trans-tree}中阻塞变量以及转换关系树的定义中得出。

从$Tail(\gamma_i)$的定义我们知道:
\begin{itemize}
	\item 如果$Length(\gamma_i)=1$ 那其中唯一的变量即为阻塞变量,

	\item 否则,$\gamma_i$ 的结构有两种可能性:$\gamma_i=\beta_1.\beta_2$或者$\gamma_i=\beta_1\llfloor\beta_2$. 在这两种情况下,阻塞的变量只能出现在右边的进程中。所以显然被阻塞的变量只能出现在最右边的儿子的标号中。
\end{proof}

更进一步分析,从$T_{\mathcal{P}}$中我们还可以的到那些潜在会增长的变量的信息。这些节点有两个或两个以上的子节点,我们称它们为\emph{分枝节点(Branching Nodes)}。这些节点在分析增长变量的过程中起到了十分重要的作用。为了分析的方便,在定义\ref{def:finite},我们为每个节点定义一个辅助的集合。

\begin{defn}[Finite 集合]\label{def:finite}
给定一个转换关系树$T_{\mathcal{P}}$,对于每个节点$N$,集合$Finite(N)\subseteq Var(\Delta)$由在树的层次归纳定义如下:
\begin{enumerate}
\item 如果$N$是$T_{\mathcal{P}}$的树根,那么$Finite(N)=Var(\Delta)$。
\item 否则,我们假设$N$的父节点是$M$:
\begin{enumerate}
\item 如果$M$不是分枝节点,那么$Finite(N)=Finite(M)$。
\item 如果$M$是分枝节点,而且$N$在$M$中未被阻塞,那么$Finite(N)=Finite(M)-\{Label(M)\}$。
\item 如果$M$是分枝节点,而且$N$在$M$中被阻塞,那么$Finite(N)=Finite(M)$。
\end{enumerate}
\end{enumerate}
\end{defn}

$Finite(N)$集合刻画了使得以$N$为根节点的子树能保持有限性质的那些$N$的后代。这一性质在引理\ref{lemma:finite}中得到了应用。

\begin{lem}\label{lemma:finite}
给定一个转换关系树$T_{\mathcal{P}}$,对于每一个节点$N$,若$Label(N)\notin Finite(N)$,那么$Label(N)$是一个增长变量。
\end{lem}

\begin{proof}
我们根据$Finite(N)$的生成过程来证明这个结论。假设$Label(N)=X$。由定义\ref{def:finite},我们可以发现从这个集合中去掉变量$X$的唯一可能是,存在另一个节点$M$,满足以下条件:
\begin{enumerate}
\item $M$是$N$在树中的一个\emph{祖先(Ancestor)}。
\item $Label(M)=X$。
\item $M$是一个分支节点且有一个子节点$O$,满足$O$在$M$中未被阻塞,且$N$由$O$被继承。
\end{enumerate}

由于$O$在$M$中未被阻塞,那么所有$O$子孙,包括$N$,不会在$M$中被阻塞。即$X\notin Tail(X)$,于是存在$\alpha$,满足$X\act{}^{*}\alpha$,$X\in Fire(\alpha)$且 $Length(\alpha)\geq 2$。

由定义\ref{def:grow-var},我们可以得到$X$是一个增长变量。
\end{proof}

为了证明定理\ref{thm:tnpa-equiv}中本章的主要结论,接下来我们针对转换关系树$T_{\mathcal{P}}$,给出关于它的概念,用来刻画它的一些细节,以方便更精细的分析。在定义\ref{def:seg}中我们将树分\emph{段(Segment)}进行研究。

\begin{defn}[树的分段 Breaking Levels and Segments]\label{def:seg}
给定一棵转换关系树$T_{\mathcal{P}}$,
\begin{itemize}
	\item 我们称那些只包含一个节点的层为\emph{分割层(Breaking Levels)},这些层中的节点为\emph{分割节点(Breaking Nodes)}。
	\item 通过分割节点,我们可以将树$T_{\mathcal{P}}$分\emph{段(Segments)}。

一个段$S$是$T_{\mathcal{P}}$中的一个子图,由分割层进行分割。每个分割节点都是一个段的\emph{起始节点(Starting Node)},直到下一个分割节点,\emph{结束节点(Ending Node)},以前的所有节点和边都属于这个段。如果不存在这样的结束节点,那么该段包含从起始节点向下的所有节点和边。

这里我们用$Seg(N)$表示一个以$N$为起始节点的段。
\end{itemize}
\end{defn}

下面我们在定义\ref{def:width}中给出另一个用来衡量的树的宽度结构的函数,我们可以通过类似的定义来衡量段的相关性质。

\begin{defn}[树的宽度 Width]\label{def:width}
给定一个以$N$为根节点的转换关系树$T_{\mathcal{P}}$,树的宽度函数$WidthT(N)$定义如下:
\begin{eqnarray*}
WidthT(N) &=& \left\{ \begin{array}{ll}
    w& \mbox{$card(Level_i)$的最大值, $i\in \mathbb{N}\cup\{0\}$}\\
    \infty &\mbox{如果不存在最大值}
    \end{array}\right.
\end{eqnarray*}
\end{defn}

其中函数$card(S)$表示集合S中互不相同的元素的个数,即集合的基数(Cardinality)。

类似的我们可以为一个段$Seg(N)$定义它的宽度函数$WidthS(N)$,或者为$T_{\mathcal{P}}$的一个以$N$为根节点的子树定义它的宽度函数$WidthT(N)$。

下面在定义\ref{def:branch-d}中,我们为转换关系树定义了一个\emph{分枝度(Branching Degree)},它的最大值被输入的tnPA的规则所限定。

\begin{defn}[Branching Degree]\label{def:branch-d}
给定一个转换关系树$T_{\mathcal{P}}$,它的\emph{分枝度(Branching Degree)},为它节点的子节点个数的最大值,即其中分枝数的最大值。我们用$\mathbf{D}$表示这个数。
\end{defn}

每个节点的分枝数由它对应被激活的转换规则所决定,所以$\mathbf{D}$的最大值由输入中进程的规则$\Delta$所限定,不会超过输入的最长的规则的长度,所以它的上界(Upper Bound)是\emph{有限(Finite)}的且可以\emph{有效计算(Effectively Computable)}的。

利用转换关系树的辅助,我们可以对进程的行为作出更细致的分析。这是一个关于进程转换路径的语法上的表示,它可以更精确地刻画其中进程表达式的“长度”,并用子节点的顺序来记录部分进程变量执行顺序的信息。通过本节的分析,我们定义了一些转换关系树上有用的结构和概念,接下来,我们就可以试图证明造成进程状态增长的原因。

\section{等价条件证明}
\label{sec:equiv-proof}

本节中我们将利用第\ref{sec:grow-prop}节中的增长变量和第\ref{sec:trans-tree}节中的转换关系树,来证明一个tnPA进程非Regular的等价条件。

首先我们需要证明引理\ref{lemma:width},它给了不包含增长变量的进程所对应的转换关系树中的所有子树的宽度一个上界。

\begin{lem}\label{lemma:width}
给定一个tnPA进程$(\alpha,\Delta)$和它的一个转换关系树$T_{\mathcal{P}}$。如果$Var(\Delta)$不包含任何增长变量,那么$T_{\mathcal{P}}$中的每个节点$N$,满足$WidthT(N)\leq D^{n-1}$,其中 $n=card(Finite(N))$。
\end{lem}

\begin{proof}
我们显然可以得到$n>0$,如果假设$n=0$,那么$Finite(N)=\emptyset$。由引理\ref{lemma:finite},由于$Label(N)\notin\Finite(N)=\emptyset$,可知$Lable(N)$是一个增长变量,和假设矛盾。所以我们之考虑$n>0$的情况。

现在我们可以利用关于$n$的归纳法进行证明:
\begin{enumerate}
\item $n=1$: 这种情况我们只要证明$WidthT(N)=\mathbf{D}^{n-1}=1$。

假设以$N$为跟的子树中有一个分枝节点$N_1$。不妨设$Finite(N)=\{X\}$,那么有$Finite(N_1)=\{X\}$。这是因为$N_1$是$N$的一个后代节点,而且$N_1$不可以是一个增长节点,根据引理\ref{lemma:finite},可知$Label(N_1)=X$。

由假设$N_1$是一个分枝节点,那么$N1$至少有一个子节点$N_1^{'}$在$N_1$中不被阻塞,于是有$Finite(N_1^{'})=\emptyset$。且有$Label(N_1^{'})=X_1\notin Finite(N_1^{'})$,则$X_1$只能是一个增长节点。产生矛盾。

这就意味着,这棵子树中不可能包含任何分枝节点,即$WidthT(N)=\mathbf{D}^{n-1}=1$。
\item $n>1$: 这种情况我们只要证明以$N$为根节点的子树中的每一个段的宽度都不会超过$\mathbf{D}^{n-1}$,其中$n=card(Finite(N))$。另$S$为该子树中一个以$N_1$为起始节点的段。我们显然有$card(Finite(N_1))\leq n$。

下面我们考虑$N_1$的四种可能情况,从而找到$WidthS(N_1)$的上界:
\begin{enumerate}
\item 若$N_1$是一个叶子节点。

那么显然$WidthS(N_1)=1$
\item 若$N_1$不是叶子节点,也不是一个分枝节点。

那么它唯一的儿子节点就是段$S$的结束节点,可知$WidthS(N_1)=1$。
\item 若$N_1$是一个分枝节点,而且$N_1$中没有子节点被阻塞。

令$N_1^{'}$为$N_1$的一个子节点,我们有$Finite(N_1^{'})=Finite(N_1)-\{Label(N_1)\}$。所以有$card(Finite(N_1^{'}))\leq n-1$。

由归纳假设,我们有$WidthT(N_1^{'})\leq \mathbf{D}^{n-2}$。$N_1$最多有$\mathbf{D}$个子节点,所以有$WidthS(N_1)\leq WidthT(N_1)\leq \mathbf{D}\times\mathbf{D}^{n-2}=\mathbf{D}^{n-1}$。
\item 若$N_1$是一个分枝节点而且有一个被阻塞的子节点$N_b$。

我们首先考虑除去$N_b$以外的那些子节点。不妨设$N_1^{'}$为这样的节点,根据上一种情况的证明,我们有$WidthT(N_1)\leq\mathbf{D}^{n-2}$

然后对于那个被阻塞的节点$N_b$,我们试图证明它在直到段$S$之前的子孙节点都不会有任何分枝节点。

如果$N_B$的子孙节点中有一个分枝节点$N_b^{'}$它属于段$S$但它不是结束节点$N_e$。由于$N_b^{'}$是分枝节点,那么他是一个激活的节点。另一方面,我们有$N_b$是被阻塞的节点,所以它的子孙节点只能在其他节点所对应的变量都消失之后才能被激活。我们可以断言$N_b^{'}$必须是$N_e$或者是它的一个子孙,这是由于$N_e$是$N_b$下面第一个分割节点。所以$N_b^{'}$不可能属于$S$,推出矛盾。

这就意味着$N_b$在$N_e$之前的所有子孙节点都不是分枝节点, 所以$N_b$对$WidthS(N_1)$的贡献只有$1$。

最后我们可以总结这种情况下$WidthS(N_1)\leq (\mathbf{D}-1)\times \mathbf{D}^{n-2}+1\leq \mathbf{D}^{n-1}$。
\end{enumerate}
于是我们有$WidthT(N)\leq \mathbf{D}^{n-1}$。
\end{enumerate}
本引理得证。
\end{proof}

有了之前的准备,我们现在就开始证明本章中的主要结论,定理\ref{thm:tnpa-equiv}。它为tnPA的$\weq_{REG}$和$\beq_{REG}$问题证明了一个充分必要条件。这是进行Regularity判定算法设计的理论基础,它确保了第\ref{chap:tnpa-alg}章中算法的正确性。

\begin{thm}[tnPA $\weq_{REG}$的等价条件]\label{thm:tnpa-equiv}
一个tnPA进程$(\alpha,\Delta)$满足$\weq_{REG}$\emph{当且仅当}$Var(\Delta)$不包含任何增长变量。
\end{thm}

\begin{proof}
我们分别证明充分性和必要性。
\begin{itemize}
	\item $\Rightarrow$ 假设tnPA进程中存在增长变量,为$X\in Var(\Delta)$。我们需要证明利用$\Delta$中的规则可以到达无限个互不弱互模拟的的状态。

有假设中$(\alpha,\Delta)$的定义,可知存在某个状态$\alpha_x$,满足$\alpha\act{}^{*}\alpha_x$,且$X\in Var(\alpha_x)$。由引理\ref{lemma:fire}可知,存在一个状态$\beta_x$,满足$\alpha\act{}^{*}\alpha_x\act{}^{*}\beta_x$,且$X\in Fire(\beta_x)$。

我们知道,互模拟的进程的Norm必然想等,由引理\ref{lemma:infi-path-3}可知,我们只需证明对于任意$k\in \mathcal{N}$,存在一个可达的状态$\beta_k$满足$\|\beta_k\|\geq k$。

这里我们选择状态$\beta_x$进行分析,令$X$为激活变量。由于$X$是一个增长变量,所以存在一个可达的状态$\beta_1^{'}$,满足$X\act{}^{*}\beta_1^{'}$,$X\in Fire(\beta_1^{'})$,且$Length(\beta_1^{'})\geq 2$。

所以存在一个状态$\beta_1$,形如$\beta_x\act{}^{*}\beta_1$,我们将其中被激活的变量$X$转换为$\beta_1^{'}$。另外,我们有$X\in Fire(\beta_1)$,因为$X\in Fire(\beta_1^{'})$,且$Fire(\beta_1^{'})\subseteq Fire(\beta_1)$。

这样我们可以重复这一过程,从$\beta_1$中构造出$\beta_2,\beta_3\ldots$,这些状态满足$X\in Fire(\beta_i)$,且$Length(\beta_{i+1})>Length(\beta_i)(i=2,3,\ldots)$。即满足$Length(\beta_k)\geq k$。

由假设,我们有$(\alpha,\Delta)$是Totally Normed,所以有对于每一个变量$Y\in Var(\beta_k)$,满足$\|Y\|\geq 1$。结合之前$Length(\beta_k)\geq k$的结论,我们就可以从Totally Normed性质得出$\|\beta_k\|\geq k$。

必要性得证。
	
	\item $\Leftarrow$ 我们只需要证明,如果一个tnPA进程$(\alpha,\Delta)$不满足$\weq_{REG}$,那么一定存在一个增长变量$X\in Var(\Delta)$。

我们基本的证明策略是找到一个不包含增长变量的进程所能产生语言上互不相同的表达式数量的上界。假设一个tnPA进程$(\alpha,\Delta)$不包含任何一个增长变量,那么我们只需要证明它是$\weq_{REG}$的。或者我们证明如果它不是$\weq_{REG}$的,那么必然存在矛盾。

由引理\ref{lemma:infi-path},这个目标等价于证明如果存在一个形如$\alpha\act{a_0}\alpha_1\act{a_1}\alpha_2\act{a_2}\ldots$的无限长度转换动作序列$\mathcal{P}$,那么存在$i\neq j$且$\alpha_i\weq \alpha_j$。

现在令$T_{\mathcal{P}}$为$\mathcal{P}$的转换关系树,我们进一步对树的宽度进行分析,以找到该进程潜在的能生成的进程表达式的长度的上限。

令$N$为$T_{\mathcal{P}}$的根节点。应用引理\ref{lemma:width},可知$WidthT(N)\leq \mathbf{D}^{n-1}$,其中$n=card(Finite(N))$,且$\mathbf{D}$为$T_{\mathcal{P}}$的分枝度。

显然,我们有$card(Finite(N))\leq card(Var(\Delta))$。那么有$Length(a_i)\leq WidthT(N)\leq \mathbf{D}^{card(Var(\Delta))-1}=m$其中,$i=1,2,\ldots$。这里$m$是一个由输入进程规则决定的常数,也就是进程从初始状态所可能生成的进程的长度由这个常数所限定。所以在$\mathcal{P}$中最多只有有限(Finite)个语言(Lexically)上互不相同的表达式。

最后,必然存在$i\neq j$,满足$\alpha_i\equiv \alpha_j$,蕴含了$\alpha_i\weq \alpha_j$。产生矛盾。

充分性得证。
\end{itemize}
综上,本定理得证。
\end{proof}

这个证明中有两处涉及到了$\weq$,首先在必要性的证明中如两个tnPA进程Norm不相等,那么它们必然不是$\weq$的。同理,它们也不是$\beq$的。

在充分性的证明中,我们用到了如果两个进程是$\equiv$那么必然是$\weq$的,同理,显然对于$\equiv$的两个进程,也满足$\beq$。综上所述,该定理对与$\beq_{REG}$一样成立,我们有推论\ref{cor:tnpa-equiv}:

\begin{cor}[tnPA $\beq_{REG}$的等价条件]\label{cor:tnpa-equiv}
一个tnPA进程$(\alpha,\Delta)$满足$\beq_{REG}$\emph{当且仅当}$Var(\Delta)$不包含任何增长变量。
\end{cor}

需要注意的是,该证明对更加一般的nPA模型并不成立。因为我们在必要性的证明中用到了tnPA中所有进程变量的Norm至少为$1$的限定条件。如果去掉这一限定,那么对于Norm为$0$的那些进程,我们就不能用进程表达式语言上的长度限制来得到Norm上的下界了。如果要克服这一问题,我们需要对进程的行为进行更精细的分析,以得到更强大的性质。

在第\ref{chap:tnpa-alg}中,我们将集中讨论如何找到一个有效的算法对该条件进行判定,并分析该算法的时间复杂度,从而找到解决tnPA的Regularity问题的具体算法。

\chapter{Totally Normed PA Regularity的算法}
\label{chap:tnpa-alg}

本章中,我们将利用第\ref{chap:tnpa-equiv}章中定理\ref{thm:tnpa-equiv}中给出的充分必要条件,设计判定tnPA的$\weq_{REG}$和$\beq_{REG}$的算法。我们将证明该问题可以在多项式时间内判定。

\section{增长变量判定算法}
\label{sec:grow-alg}

在定理\ref{thm:tnpa-equiv}中,我们给出了tnPA的Regularity性质的一个充分必要条件,即增长变量的性质。本节我们对一个给定的tnPA进程$(\alpha,\Delta)$,将给出一个多项式时间算法。由我们PA进程的定义\ref{def:pa}中的假设,对于每个进程变量$X\in Var(\Delta)$,必定存在一个可达的状态,使他可以被激活。所以,我们只需要对输入中给定的规则,判断其中是否包含一个增长变量。

我们现在证明引理\ref{lemma:grow-dec}。

\begin{lem}[增长变量的可判定性 Decdability for Growing Variables]\label{lemma:grow-dec}
给定一个tnPA进程$(\alpha,\Delta)$,其中是否存在一个增长变量$X\in Var(\Delta)$,是可判定的。
\end{lem}

\begin{proof}
见算法\ref{alg:grow} \textsl{GROW}。
\end{proof}

\begin{algorithm}[htbp]
\caption{GROW}
\label{alg:grow}
\begin{algorithmic}[1]
\Statex \textbf{Grow$(\Delta)$ }
\For {all $X\in Var(\Delta)$}
    \State $S\leftarrow \{X\}$
    \While {$S\neq Transition1(S)$}
        \State $S\leftarrow Transition1(S)$
    \EndWhile
    \State $S\leftarrow Transition2(S)$
    \While {$S\neq Transition1(S)$}
        \State $S\leftarrow Transition1(S)$
    \EndWhile
    \If {$X\in S$}
        \State \Return TRUE
    \EndIf
\EndFor
\State \Return FALSE

\end{algorithmic}
\end{algorithm}

其中$Transition1$和$Transition2$分别由算法\ref{alg:tran1} \textsl{TRANSITION1},和算法\ref{alg:tran2} \textsl{TRANSITION2}计算:

\begin{algorithm}[htbp]
\caption{TRANSITION1}
\label{alg:tran1}
\begin{algorithmic}[1]
\Statex \textbf{Transition1$(S)$}
\For {all $Y\in Var(\Delta)$}
    \If {$\exists X\in S$ and a rule $X\act{a}\alpha \in \Delta$ and $Y\in Var(\alpha)$ }
        \State $S\leftarrow S\cup \{Y\}$
    \EndIf
\EndFor
\State \Return $S$
\end{algorithmic}
\end{algorithm}

\begin{algorithm}[htbp]
\caption{TRANSITION2}
\label{alg:tran2}
\begin{algorithmic}[1]
\Statex \textbf{Transition2$(S)$}
\State $S^{'}\leftarrow \emptyset$
\For {all $Y\in Var(\Delta)$}
    \If {$\exists X\in S$ and a rule $X\act{a}\alpha \in \Delta$ with $Y\in Var(\alpha)$, $Length(\alpha)\geq 2$ and $Y\notin Tail(\alpha)$}
        \State $S^{'}\leftarrow S^{'}\cup \{Y\}$
    \EndIf
\EndFor
\State \Return $S^{'}$
\end{algorithmic}
\end{algorithm}
\end{proof}

\begin{enumerate}
\item 算法\ref{alg:grow} \textsl{GROW}的第一步中,对于函数\textsl{Transition1(S)}的迭代调用计算了关系$(X,Y)$的自反传递闭包,其中$(X,Y)$满足存在一条规则$X\act{a}\alpha\in\Delta$,且$Y\in\ Var(\alpha)$。

得到的集合恰好包含了由$X$生成的所有可能被激活的变量。

\item 下一步中,对\textsl{Transition2(S)}的调用,计算了关系$(X,Y)$。$(X,Y)$满足存在一条规则$X\act{a}\alpha\in\Delta$,且有$Y\in\ Var(\alpha)$,$Y\notin Tail(\alpha)$,以及$Lenth(\alpha)\geq 2$。

这一步结束后,集合中所有变量都是被激活的,且集合包含了所有满足不被阻塞,而且经过一步转换后至少还存在另一个变量的变量。


\item 算法的最后一步,我们再次迭代调用函数\textsl{Transition1(S)},它将生成所有包含之前性质的变量的集合。然后我们可以检测变量$X$是否还在所得的集合中。

\end{enumerate}

如果结果是肯定的,那么变量$X$是一个增长变量。

另一方面,如果一个变量是增长的,那么由增长变量的定义\ref{def:grow-var},由于我们检测了所有变量,所以它的增长性一定能被该算法检测出来。

\section{时间复杂度分析}
\label{sec:complexity}

在第\ref{sec:grow-alg}节中,我们给出了判定增长变量的\textsl{GROW}算法以及其正确性(Soundness)和完备性(Completeness)的分析和证明。在本节中,我们将进一步地分析它的时间复杂度,证明其复杂度是\emph{多项式时间的(Polynomial Time)}的。

我们知道,在现实的计算中,通常可以处理(Feasible)的计算必须是多项式时间的。如果超过了这个限制范围,比如是\emph{指数时间(Exponential Time)}的,那么计算的时间将会随着问题规模的增长呈指数增长,这种开销的增长是我们所不能忍受的。为了能让本文中的算法有更好的实际应用价值,我们必须对其做复杂度分析。

在引理\ref{lemma:complexity}中,我们给出了算法\textsl{GROW}的时间复杂度。

\begin{lem}[算法时间复杂度 Complexity of the Algorithm]\label{lemma:complexity}
该判定算法\ref{alg:grow} \textsl{GROW}的时间复杂度为$\mathcal{O}(n^3+mn)$,其中$n$为输入的规则的个数,$m$为其中最长规则的长度。
\end{lem}

\begin{proof}

假设输入的进程$(\alpha,\Delta)$中,一共有$n$条规则,其中最长规则的长度为$m$。

那么最多有$n$个互不相同的进程变量。
\begin{enumerate}

\item 对于每条规则$X\act{a}\alpha\in\Delta$中的进程表达式$\alpha$,我们首先计算$Tail(\alpha)$。这一步的时间复杂度为$\mathcal{O}(mn)$。因为最多只有个$n$集合,对于每个集合,我们最多需要迭代$m$次。

\item 同样,我们也可以对于每个变量,对两个\textsl{TRANSITION}函数进行预处理。这一步,我们个一对每个变量建立一个表,保存满足条件的变量。这一步计算的复杂度为$\mathcal{O}(mn)$,因为我们一共需要处理最多$n$条规则,长度最多为$m$。

\item 接下来我们继续对两个\textsl{TRANSITION}函数进行简单的分析。最大的循环次数,对于任何变量,做多有$n$条不同的规则,所以最多为$n$。对于每条规则,我们对条件的检测需要$\mathcal{O}(1)$的时间,所以这两个函数的时间复杂度为$\mathcal{O}(n)$。

\item 最后,我们考虑主函数\textsl{GROW}。外层循环需要最多执行$n$次,而内层循环在最多执行$n$次后也会到达不动点(Fixpoint)。

\end{enumerate}
综上,算法总共的时间复杂度为$\mathcal{O}(mn)+\mathcal{O}(n^{2})\cdot \mathcal{O}(n)=\mathcal{O}(n^3+mn)$。
\end{proof}

由定理\ref{thm:tnpa-equiv}和引理\ref{lemma:grow-dec},引理\ref{lemma:complexity}。我们可以得到tnPA的$\weq_{REG}$问题的主要结论。

\begin{thm}[tnPA的$\weq_{REG}$]\label{thm:tnpa-wreg}
给定一个tnPA进程$(\alpha,\Delta)$,它的$\weq_{REG}$问题可以在多项式时间内被判定。时间复杂度为$\mathcal{O}(n^3+mn)$,其中$n$为输入的规则的个数,$m$为其中最长规则的长度。
\end{thm}

由推论\ref{cor:tnpa-equiv}我们可知,该结论对于tnPA的$\beq_{REG}$同样成立。我们有推论\ref{cor:tnpa-breg}

\begin{cor}[tnPA的$\beq_{REG}$]\label{cor:tnpa-breg}
给定一个tnPA进程$(\alpha,\Delta)$,它的$\beq_{REG}$问题可以在多项式时间内被判定。时间复杂度为$\mathcal{O}(n^3+mn)$,其中$n$为输入的规则的个数,$m$为其中最长规则的长度。
\end{cor}

有第\ref{sec:prs}节中介绍的PRS层次中的模型包含关系可知,BPA和BPP都是PA的子模型,所以该结论在tnBPA和tnBPP上都成立。我们有推论\ref{cor:tnbpa-tnbpp}

\begin{cor}[tnBPA和tnBPP的$\weq{REG}$和$\beq_{REG}$]\label{cor:tnbpa-tnbpp}
给定一个tnBPA或tnBPP进程$(\alpha,\Delta)$,它的$\weq_{REG}$和$\beq_{REG}$问题均可以在多项式时间内被判定。时间复杂度为$\mathcal{O}(n^3+mn)$,其中$n$为输入的规则的个数,$m$为其中最长规则的长度。
\end{cor}

到此为止,我们就完成了对Totally Normed PA相关的$\weq_{REG}$和$\beq_{REG}$问题可判定性的研究,并证明了其多项式的时间复杂度。

\chapter{后续问题研究讨论}
\label{chap:future}

%%==================================================
%% conclusion.tex for SJTU Master Thesis
%% based on CASthesis
%% modified by wei.jianwen@gmail.com
%% version: 0.3a
%% Encoding: UTF-8
%% last update: Dec 5th, 2010
%%==================================================

\chapter*{全文总结\markboth{全文总结}{}}
\addcontentsline{toc}{chapter}{全文总结}

这里是全文总结内容。

 %% 全文总结


%%%%%%%%%%%%%%%%%%%%%%%%%%%%%% 
%% 附录(章节编号重新计算,使用字母进行编号)
%%%%%%%%%%%%%%%%%%%%%%%%%%%%%% 
\appendix

% 附录中编号形式是"A-1"的样子
\renewcommand\theequation{\Alph{chapter}--\arabic{equation}}
\renewcommand\thefigure{\Alph{chapter}--\arabic{figure}}
\renewcommand\thetable{\Alph{chapter}--\arabic{table}}

%%%==================================================
%% app1.tex for SJTU Master Thesis
%% based on CASthesis
%% modified by wei.jianwen@gmail.com
%% version: 0.3a
%% Encoding: UTF-8
%% last update: Dec 5th, 2010
%%==================================================

\chapter{模板更新记录}
\label{chap:updatelog}

\textbf{2013年5月26日} v0.5.3发布,更正subsubsection格式错误,这个错误导致如"1.1 小结"这样的标题没有被正确加粗。

\textbf{2012年12月27日} v0.5.2发布,更正拼写错误:从``个人建立''更正为``个人简历''。在diss.tex加入ack.tex,更名后忘了引用。

\textbf{2012年12月21日} v0.5.1发布,在 \LaTeX 命令和中文字符之间留了空格,在Makefile中增加release功能。

\textbf{2012年12月5日} v0.5发布,修改说明文件的措辞,更正Makefile文件,使用metalog宏包替换xltxtra宏包,使用mathtools宏包替换amsmath宏包,移除了所有CJKtilde(\verb+~+)符号。

\textbf{2012年5月30日} v0.4发布,包含交大学士、硕士、博士学位论文模板。模板在\href{https://github.com/weijianwen/sjtu-thesis-template-latex}{github}上管理和更新。

\textbf{2010年12月5日} v0.3a发布,移植到 \XeTeX/\LaTeX 上。

\textbf{2009年12月25日} v0.2a发布,模板由CASthesis改名为sjtumaster。在diss.tex中可以方便地改变正文字号、切换但双面打印。增加了不编号的一章“全文总结”。
添加了可伸缩符号(等号、箭头)的例子,增加了长标题换行的例子。

\textbf{2009年11月20日} v0.1c发布,增加了Linux下使用ctex宏包的注意事项、.bib条目的规范要求,
修正了ctexbook与listings共同使用时的断页错误。

\textbf{2009年11月13日} v0.1b发布,完善了模板使用说明,增加了定理环境、并列子图、三线表格的例子。

\textbf{2009年11月12日} 上海交通大学硕士学位论文 \LaTeX 模板发布,版本0.1a。

 % 更新记录
%%% app2.tex for SJTU Master Thesis
%% based on CASthesis
%% modified by wei.jianwen@gmail.com
%% version: 0.3a
%% Encoding: UTF-8
%% last update: Dec 5th, 2010
%%==================================================

\chapter{Maxwell Equations}

选择二维情况,有如下的偏振矢量
\begin{subequations}
  \begin{eqnarray}
    {\bf E}&=&E_z(r,\theta)\hat{\bf z} \\
    {\bf H}&=&H_r(r,\theta))\hat{ \bf r}+H_\theta(r,\theta)\hat{\bm
      \theta}
  \end{eqnarray}
\end{subequations}
对上式求旋度
\begin{subequations}
  \begin{eqnarray}
    \nabla\times{\bf E}&=&\frac{1}{r}\frac{\partial E_z}{\partial\theta}{\hat{\bf r}}-\frac{\partial E_z}{\partial r}{\hat{\bm\theta}}\\
    \nabla\times{\bf H}&=&\left[\frac{1}{r}\frac{\partial}{\partial
        r}(rH_\theta)-\frac{1}{r}\frac{\partial
        H_r}{\partial\theta}\right]{\hat{\bf z}}
  \end{eqnarray}
\end{subequations}
因为在柱坐标系下,$\overline{\overline\mu}$是对角的,所以Maxwell方程组中电场$\bf
E$的旋度
\begin{subequations}
  \begin{eqnarray}
    &&\nabla\times{\bf E}=\mathbf{i}\omega{\bf B} \\
    &&\frac{1}{r}\frac{\partial E_z}{\partial\theta}{\hat{\bf
        r}}-\frac{\partial E_z}{\partial
      r}{\hat{\bm\theta}}=\mathbf{i}\omega\mu_rH_r{\hat{\bf r}}+\mathbf{i}\omega\mu_\theta
    H_\theta{\hat{\bm\theta}}
  \end{eqnarray}
\end{subequations}
所以$\bf H$的各个分量可以写为:
\begin{subequations}
  \begin{eqnarray}
    H_r=\frac{1}{\mathbf{i}\omega\mu_r}\frac{1}{r}\frac{\partial
      E_z}{\partial\theta } \\
    H_\theta=-\frac{1}{\mathbf{i}\omega\mu_\theta}\frac{\partial E_z}{\partial r}
  \end{eqnarray}
\end{subequations}
同样地,在柱坐标系下,$\overline{\overline\epsilon}$是对角的,所以Maxwell方程组中磁场$\bf
H$的旋度
\begin{subequations}
  \begin{eqnarray}
    &&\nabla\times{\bf H}=-\mathbf{i}\omega{\bf D}\\
    &&\left[\frac{1}{r}\frac{\partial}{\partial
        r}(rH_\theta)-\frac{1}{r}\frac{\partial
        H_r}{\partial\theta}\right]{\hat{\bf
        z}}=-\mathbf{i}\omega{\overline{\overline\epsilon}}{\bf
      E}=-\mathbf{i}\omega\epsilon_zE_z{\hat{\bf z}} \\
    &&\frac{1}{r}\frac{\partial}{\partial
      r}(rH_\theta)-\frac{1}{r}\frac{\partial
      H_r}{\partial\theta}=-\mathbf{i}\omega\epsilon_zE_z
  \end{eqnarray}
\end{subequations}
由此我们可以得到关于$E_z$的波函数方程:
\begin{eqnarray}
  \frac{1}{\mu_\theta\epsilon_z}\frac{1}{r}\frac{\partial}{\partial r}
  \left(r\frac{\partial E_z}{\partial r}\right)+
  \frac{1}{\mu_r\epsilon_z}\frac{1}{r^2}\frac{\partial^2E_z}{\partial\theta^2}
  +\omega^2 E_z=0
\end{eqnarray}
 % 麦克斯韦方程
% \include{body/app3}


%%%%%%%%%%%%%%%%%%%%%%%%%%%%%% 
%% 文后(无章节编号)
%%%%%%%%%%%%%%%%%%%%%%%%%%%%%% 
\backmatter

% 参考文献
% 使用 BibTeX
% 包含参考文献文件.bib
\bibliography{reference/chap1,reference/chap2,reference/ref}

%% 个人简历(硕士学位论文没有个人简历要求)
% %%==================================================
%% resume.tex for SJTU Master Thesis
%% based on CASthesis
%% modified by wei.jianwen@gmail.com
%% version: 0.3a
%% Encoding: UTF-8
%% last update: Dec 5th, 2010
%%==================================================

\begin{resume}

\begin{resumesection}{基本情况}
xxx,男,上海人,1985 年~12 月出生,未婚,
上海交通大学物理系在读博士研究生。
\end{resumesection}

\begin{resumelist}{教育状况}
XXXX 年~9 月至~XXXX 年~7 月,上海交通大学, 本科,专业:XXXX

XXXX 年~9 月至~XXXX 年~7 月,上海交通大学, 硕士研究生,专业:XXXX

XXXX 年~9 月至~XXXX 年~7 月,上海交通大学,
博士研究生(提前攻读博士),专业:XXXX
\end{resumelist}

\begin{resumelist}{工作经历}
无。
\end{resumelist}

\begin{resumelist}{研究兴趣}
XXXXXXX。
\end{resumelist}

\begin{resumelist}{联系方式}
通讯地址:上海市闵行区东川路800号,上海交通大学物理系

邮编:200240

E-mail: abcde@sjtu.edu.cn
\end{resumelist}

\end{resume}


% 致谢
%%==================================================
%% thanks.tex for SJTU Master Thesis
%% based on CASthesis
%% modified by wei.jianwen@gmail.com
%% version: 0.3a
%% Encoding: UTF-8
%% last update: Dec 5th, 2010
%%==================================================

\begin{thanks}

  感谢所有测试和使用交大硕士学位论文 \LaTeX 模板的同学!

  感谢那位最先制作出博士学位论文 \LaTeX 模板的交大物理系同学!

  感谢~William Wang~同学对模板移植做出的巨大贡献!

\end{thanks}


% 发表文章目录
%%==================================================
%% pub.tex for SJTU Master Thesis
%% based on CASthesis
%% modified by wei.jianwen@gmail.com
%% version: 0.3a
%% Encoding: UTF-8
%% last update: Dec 5th, 2010
%%==================================================

\begin{publications}{99}

    \item\textsc{Chen H, Chan C~T}. {Acoustic cloaking in three dimensions using acoustic metamaterials}[J].
      Applied Physics Letters, 2007, 91:183518.

    \item\textsc{Chen H, Wu B~I, Zhang B}, et al. {Electromagnetic Wave Interactions with a Metamaterial Cloak}[J].
      Physical Review Letters, 2007, 99(6):63903.
    
\end{publications}


% 参与项目列表
%%==================================================
%% projects.tex for SJTU Master Thesis
%% based on CASthesis
%% modified by wei.jianwen@gmail.com
%% version: 0.3a
%% Encoding: UTF-8
%% last update: Dec 5th, 2010
%%==================================================

\begin{projects}{99}

    \item 973项目“XXX”
    \item 自然基金项目“XXX”
    \item 国防项目“XXX”
    
\end{projects}


\end{document}
