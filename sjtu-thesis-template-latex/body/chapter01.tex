%%==========================
%% chapter01.tex for SJTU Master Thesis
%% based on CASthesis
%% modified by wei.jianwen@gmail.com
%% version: 0.3a
%% Encoding: UTF-8
%% last update: Dec 5th, 2010
%%==================================================

%\bibliographystyle{sjtu2} %[此处用于每章都生产参考文献]
\chapter{绪论}
\label{chap: 1}

\section{研究背景}
\label{sec:background}

在计算机科学理论的发展过程中,许多不同的计算模型被相继提出。最初的计算模型所定义的都是串行计算。例如图灵机(Turing Machine)\cite{Turing1936}, $\lambda$-演算($\lambda$-Calculus)\cite{Church1985}, 递归函数(Recursive Function)\cite{Rogers1967}等。随着并行化计算的发展,理论计算机科学家提出了并行化的计算模型并进行了深入的研究。其中比较有代表性的是由R.Milner提出的Communication Concurrency System (CCS)\cite{Milner1989}。该模型利用进程演算的方法对可以实现并发和交互的进程模型进行了刻画。

在这些计算模型的研究中,涉及到一个计算机科学中十分重要的领域:形式化验证(Formal Verification)。而这些计算模型,大多数都可以描述无限状态系统。对无限状态系统的形式化验证(Verification on Infinite State Systems),是现今理论计算机科学中的一个热门的研究方向,它包含了一系列具有重要意义的研究课题。这些问题往往可以与可计算理论(Computability), 计算复杂性理论(Computational Complexity),算法(Algorithm)等领域中的一些经典问题相联系起来,从而得出许多振奋人心的可计算性或复杂性结论。

为了对这些模型进行验证,我们需要选择合适的等价关系(Equivalence Relation)。人们最初研究的等价关系是语言等价(Language Equivalence),然而即使对于上下文无关语言(Context Free Language),其语言等价也是不可判定的\cite{Hopcroft1979}。而不可判定的等价关系从验证角度来说是用处不大的。于是许多基于观测理论(Observation Theory) 中互模拟(Bisimulation) 概念的等价关系被提出,最早的是由 Park 提出的强互模拟(Strong Bisimulation)\cite{Park1981}。随后,为了区别系统中内部动作和外部动作,Milner引入了$\tau$动作表示系统内部的转换,并且定义了弱互模拟(Weak Bisimulation)\cite{Milner1989}。为了更加精细地区分$\tau$动作对系统状态的影响,van Glabbeek和 Weijland提出了Branching 互模拟(Branching Bisimulation)\cite{Glabbeek1996}。这些基于互模拟的等价关系,是研究验证问题的理论基础。

在研究无限状态系统时,系统的模型可以使用进程(Process)进行表示。对于各种不同种类的无限状态系统,都可以利用一个统一的进程代数模型进行表示,即进程重写系统(Process Rewrite System, 简称PRS)\cite{Mayr2000}。PRS是一个具有一般性的进程模型,它提供不同模型的关于强互模拟的表达能力的层次结构。许多常见的进程代数模型都可以在这个层次结构中找到。

而对于一个具体的模型所描述的系统,确定需要研究的等价关系后,我们所关心的问题主要分3类,分别是关于该等价关系的等价性判定(Equivalence Checking),与某个给定的有限状态系统的等价性判定(Finiteness)和是否存在一个有限状态系统与该系统等价(Regularity)。这3类问题在本质上有着一定的联系,但是在解决的方法和难度上却有着区别。本文将重点对第三类问题,即有限性问题(Regularity Problem)进行讨论。

\section{国内外研究现状}
\label{sec:state of the art}

无限状态系统的验证近年来一直是理论计算机科学中的一个十分活跃的研究领域。在这方面最早的可判定性结论是由Baeten, Bergstra和 Klop证明的上下文无关语法关于强互模拟等价的可判定性\cite{Baeten1993}。 这一结论引发了对于各种不同无限状态系统的等价性判定的研究,许多问题的可判定性和复杂性结论被提出并得到了证明。我们可以找到许多相关的调查研究\cite{Burkart2001, Kucera2006, Moller2004, Srba2002}。

这些研究所涉及到的大多数模型都可以由进程重写系统所产生\cite{Mayr2000}。这些研究都极大的提升了我们对于一些经典无限状态系统模型的认识,包括基本进程代数(Basic Process Algebra, 简称BPA)\cite{Bergstra1985},基本并行进程(Basic Parallel Process, 简称BPP)
