%%==================================================

\chapter*{全文总结\markboth{全文总结}{}}
\addcontentsline{toc}{chapter}{全文总结}

本文对一类无限状态系统的验证问题进行了探究。在进程重写系统(PRS)的模型框架下,对进程系统的有限性(Regularity)问题进行了研究。我们对于这一类问题用到的一般性的技术和现有的结论进行了深入的总结和讨论。对于一类具体的模型PA和考虑内部动作的互模拟等价关系,进行了Regularity问题的讨论,并提出了解决问题的算法,以及复杂度分析。最后,我们今后可以希望能解决的相关问题,进行了归类,讨论并提出了可行的技术路线和对结论的猜想。

Regularity问题所研究的是进程的有限性,即是否存在一个有限状态系统(FS)和给定的进程等价。在PRS结构中,很多模型Regularity问题都有了相关的结论。在本文中我们对这些结论进行了系统的总结和分类,并得到了一些十分有用的技术上的引理和推论。

进程代数(PA)是在PRS层次结构中的一个非常重要的模型,它的规则及允许串行连结符,也允许并行连结符,是一个十分一般化的上下文无关的模型。基本的串行(BPA)和并行(BPP)的进程模型,都被证明是它的子模型。

现阶段关于进程重写系统的Regularity问题的可判定性与算法的结论,基本都是关于强互模拟关系$\seq$的,然而在实际问题中,考虑进程内部的$\tau$动作的弱互模拟$\weq$和Branching互模拟经常会用到。然而,内部动作的引入,也会给研究进程的行为带来极大的复杂性和困难性。

所以,本文中为了解决这些困难,利用了进程Totally Normed的限制条件,找到一个能使进程产生无限状态的子结构。我们称这类子结构为增长的进程变量。我们利用进程的转换关系树对进程行为进行分析,证明了该子结构的存在就是进程产生无限状态的充分必要条件。我们所给出的Totally Normed PA的$\weq_{REG}$和$\beq_{REG}$问题的判定算法都被证明是多项式时间的。其时间复杂度为$\mathscr{O}(n^3+mn)$,其中$n$是输入的进程系统规则条数,而$m$是其中最长规则的长度。

最后,我们对今后的工作进行了展望,提出了几类现阶段可以尝试解决的问题。包括nBPP的$\beq_{REG}$,tnPN的$\weq_{REG}$和$\beq_{REG}$,以及PRS上Regularity问题的一些下界的证明。我们对这些问题进行了初步的分析,并给出了一些技术上的引理和结论上的猜想。为今后的工作提供了思路,打下了基础。

总的来说,本文的贡献包括如下几点工作:

\begin{enumerate}
	\item 总结PRS上Regularity问题的研究现状,分析了现有结论和技术,提出了一些推论和引理。
	\item 解决了tnPA的$\weq_{REG}$和$\beq_{REG}$问题,并证明了其判定算法的时间复杂度为$\mathscr{O}(n^3+mn)$。其中$m$和$n$都是和输入相关的常数,所以该算法是多项式时间的。
	\item 对于nBPP的$\beq_{REG}$,以及tnPN的$\weq_{REG}$和$\beq_{REG}$问题进行了讨论,提出了解决的思路和一些引理。
	\item 对于PRS上Regularity问题的下界进行了总结和讨论,提出了一些希望能证明的问题和可能的结论。
\end{enumerate}

同时,本文的未来工作除了理论上的进一步深入研究,还希望能将文中提出的多项式时间算法完成系统化的实现,在实际应用场景中发挥它的作用。
