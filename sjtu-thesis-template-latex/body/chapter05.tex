\chapter{Totally Normed PA Regularity的算法}
\label{chap:tnpa-alg}

本章中,我们将利用第\ref{chap:tnpa-equiv}章中定理\ref{thm:tnpa-equiv}中给出的充分必要条件,设计判定tnPA的$\weq_{REG}$和$\beq_{REG}$的算法。我们将证明该问题可以在多项式时间内判定。

\section{增长变量判定算法}
\label{sec:grow-alg}

在定理\ref{thm:tnpa-equiv}中,我们给出了tnPA的Regularity性质的一个充分必要条件,即增长变量的性质。本节我们对一个给定的tnPA进程$(\alpha,\Delta)$,将给出一个多项式时间算法。由我们PA进程的定义\ref{def:pa}中的假设,对于每个进程变量$X\in Var(\Delta)$,必定存在一个可达的状态,使他可以被激活。所以,我们只需要对输入中给定的规则,判断其中是否包含一个增长变量。

我们现在证明引理\ref{lemma:grow-dec}。

\begin{lem}[增长变量的可判定性 Decdability for Growing Variables]\label{lemma:grow-dec}
给定一个tnPA进程$(\alpha,\Delta)$,其中是否存在一个增长变量$X\in Var(\Delta)$,是可判定的。
\end{lem}

\begin{proof}
见算法\ref{alg:grow} \textsl{GROW}。
\end{proof}

\begin{algorithm}[htbp]
\caption{GROW}
\label{alg:grow}
\begin{algorithmic}[1]
\Statex \textbf{Grow$(\Delta)$ }
\For {all $X\in Var(\Delta)$}
    \State $S\leftarrow \{X\}$
    \While {$S\neq Transition1(S)$}
        \State $S\leftarrow Transition1(S)$
    \EndWhile
    \State $S\leftarrow Transition2(S)$
    \While {$S\neq Transition1(S)$}
        \State $S\leftarrow Transition1(S)$
    \EndWhile
    \If {$X\in S$}
        \State \Return TRUE
    \EndIf
\EndFor
\State \Return FALSE

\end{algorithmic}
\end{algorithm}

其中$Transition1$和$Transition2$分别由算法\ref{alg:tran1} \textsl{TRANSITION1},和算法\ref{alg:tran2} \textsl{TRANSITION2}计算:

\begin{algorithm}[htbp]
\caption{TRANSITION1}
\label{alg:tran1}
\begin{algorithmic}[1]
\Statex \textbf{Transition1$(S)$}
\For {all $Y\in Var(\Delta)$}
    \If {$\exists X\in S$ and a rule $X\act{a}\alpha \in \Delta$ and $Y\in Var(\alpha)$ }
        \State $S\leftarrow S\cup \{Y\}$
    \EndIf
\EndFor
\State \Return $S$
\end{algorithmic}
\end{algorithm}

\begin{algorithm}[htbp]
\caption{TRANSITION2}
\label{alg:tran2}
\begin{algorithmic}[1]
\Statex \textbf{Transition2$(S)$}
\State $S^{'}\leftarrow \emptyset$
\For {all $Y\in Var(\Delta)$}
    \If {$\exists X\in S$ and a rule $X\act{a}\alpha \in \Delta$ with $Y\in Var(\alpha)$, $Length(\alpha)\geq 2$ and $Y\notin Tail(\alpha)$}
        \State $S^{'}\leftarrow S^{'}\cup \{Y\}$
    \EndIf
\EndFor
\State \Return $S^{'}$
\end{algorithmic}
\end{algorithm}
\end{proof}

\begin{enumerate}
\item 算法\ref{alg:grow} \textsl{GROW}的第一步中,对于函数\textsl{Transition1(S)}的迭代调用计算了关系$(X,Y)$的自反传递闭包,其中$(X,Y)$满足存在一条规则$X\act{a}\alpha\in\Delta$,且$Y\in\ Var(\alpha)$。

得到的集合恰好包含了由$X$生成的所有可能被激活的变量。

\item 下一步中,对\textsl{Transition2(S)}的调用,计算了关系$(X,Y)$。$(X,Y)$满足存在一条规则$X\act{a}\alpha\in\Delta$,且有$Y\in\ Var(\alpha)$,$Y\notin Tail(\alpha)$,以及$Lenth(\alpha)\geq 2$。

这一步结束后,集合中所有变量都是被激活的,且集合包含了所有满足不被阻塞,而且经过一步转换后至少还存在另一个变量的变量。


\item 算法的最后一步,我们再次迭代调用函数\textsl{Transition1(S)},它将生成所有包含之前性质的变量的集合。然后我们可以检测变量$X$是否还在所得的集合中。

\end{enumerate}

如果结果是肯定的,那么变量$X$是一个增长变量。

另一方面,如果一个变量是增长的,那么由增长变量的定义\ref{def:grow-var},由于我们检测了所有变量,所以它的增长性一定能被该算法检测出来。

\section{时间复杂度分析}
\label{sec:complexity}

在引理\ref{lemma:complexity}中,我们给出了算法的时间复杂度。

\begin{lem}[算法时间复杂度 Complexity of the Algorithm]\label{lemma:complexity}
该判定算法\ref{alg:grow} \textsl{GROW}的时间复杂度为$\mathcal{O}(n^3+mn)$,其中$n$为输入的规则的个数,$m$为其中最长规则的长度。
\end{lem}

\begin{proof}

假设输入的进程$(\alpha,\Delta)$中,一共有$n$条规则,其中最长规则的长度为$m$。

那么最多有$n$个互不相同的进程变量。
\begin{enumerate}

\item 对于每条规则$X\act{a}\alpha\in\Delta$中的进程表达式$\alpha$,我们首先计算$Tail(\alpha)$。这一步的时间复杂度为$\mathcal{O}(mn)$。因为最多只有个$n$集合,对于每个集合,我们最多需要迭代$m$次。

\item 同样,我们也可以对于每个变量,对两个\textsl{TRANSITION}函数进行预处理。这一步,我们个一对每个变量建立一个表,保存满足条件的变量。这一步计算的复杂度为$\mathcal{O}(mn)$,因为我们一共需要处理最多$n$条规则,长度最多为$m$。

\item 接下来我们继续对两个\textsl{TRANSITION}函数进行简单的分析。最大的循环次数,对于任何变量,做多有$n$条不同的规则,所以最多为$n$。对于每条规则,我们对条件的检测需要$\mathcal{O}(1)$的时间,所以这两个函数的时间复杂度为$\mathcal{O}(n)$。

\item 最后,我们考虑主函数\textsl{GROW}。外层循环需要最多执行$n$次,而内层循环在最多执行$n$次后也会到达不动点(Fixpoint)。

\end{enumerate}
综上,算法总共的时间复杂度为$\mathcal{O}(mn)+\mathcal{O}(n^{2})\cdot \mathcal{O}(n)=\mathcal{O}(n^3+mn)$。
\end{proof}

由定理\ref{thm:tnpa-equiv}和引理\ref{lemma:grow-dec},引理\ref{lemma:complexity}。我们可以得到tnPA的$\weq_{REG}$问题的主要结论。

\begin{thm}[tnPA的$\weq_{REG}$]\label{thm:tnpa-wreg}
给定一个tnPA进程$(\alpha,\Delta)$,它的$\weq_{REG}$问题可以在多项式时间内被判定。时间复杂度为$\mathcal{O}(n^3+mn)$,其中$n$为输入的规则的个数,$m$为其中最长规则的长度。
\end{thm}

由推论\ref{cor:tnpa-equiv}我们可知,该结论对于tnPA的$\beq_{REG}$同样成立。我们有推论\ref{cor:tnpa-breg}

\begin{cor}[tnPA的$\beq_{REG}$]\label{cor:tnpa-breg}
给定一个tnPA进程$(\alpha,\Delta)$,它的$\beq_{REG}$问题可以在多项式时间内被判定。时间复杂度为$\mathcal{O}(n^3+mn)$,其中$n$为输入的规则的个数,$m$为其中最长规则的长度。
\end{cor}

有第\ref{sec:prs}节中介绍的PRS层次中的模型包含关系可知,BPA和BPP都是PA的子模型,所以该结论在tnBPA和tnBPP上都成立。我们有推论\ref{cor:tnbpa-tnbpp}

\begin{cor}[tnBPA和tnBPP的$\weq{REG}$和$\beq_{REG}$]\label{cor:tnbpa-tnbpp}
给定一个tnBPA或tnBPP进程$(\alpha,\Delta)$,它的$\weq_{REG}$和$\beq_{REG}$问题均可以在多项式时间内被判定。时间复杂度为$\mathcal{O}(n^3+mn)$,其中$n$为输入的规则的个数,$m$为其中最长规则的长度。
\end{cor}


