%%==================================================
\chapter{背景知识}
\label{chap:pre}

在这一部分,我们首先介绍一下研究PRS上Regularity问题的背景知识。包括相关的模型和一些有趣的互模拟等价关系,以及研究无限状态系统验证的几个基本问题和目标。

\section{进程重写系统 PRS}
\label{sec:prs}

\subsection{基本定义}
\label{subsec:basic-def}

进程重写系统(PRS)是一个可以用来刻画进程模型语义的一般系统,这一部分我们会给出关于进程重写系统的一些基本定义。\cite{Mayr2000}

在此之前,我们可以分析一个进程代数中的例子\cite{Milner1989},以下定义了一个计数器(Counter Machine)的\emph{需求(Specification)}。
\begin{eqnarray*}
C_{0} &=& zero.C_{0}+inc.C_{1}, \\
C_{i+1} &=& dec.C_{i}+inc.C_{i+2},\ \mathrm{where}\ i\ge0.
\end{eqnarray*}
下面是Busi, Gabbrielli 和 Zavattaro 给出的\emph{实现(Implementation)}\cite{Busi2003}:
\begin{eqnarray*}
Counter &=& zero.Counter+inc.(d)(O\,|\,d.Counter), \\
O &=& dec.\overline{d}+inc.(e)(E\,|\,e.O), \\
E &=& dec.\overline{e}+inc.(d)(O\,|\,d.E).
\end{eqnarray*}
用BPA来\emph{编码(Programming)}就是:
\[
Z \stackrel{inc}{\longrightarrow} XZ, \ \ \
Z \stackrel{zero}{\longrightarrow} Z, \ \ \
X \stackrel{inc}{\longrightarrow} XX, \ \ \
X \stackrel{dec}{\longrightarrow} \epsilon.
\]
通过这个例子,我们可以直观的看出PRS中的BPA模型可以编码一个计数器。当然我们也可以通过更加复杂的编码来实现更加复杂工作的验证。我们下面给出PRS语法的定义和语义的规则。

\begin{defn}[进程项 Process Term]\label{def:process-term}
令$Act=\{a,b,\ldots\}$是一个\emph{原子动作(atomic actions)}的集合;$Const=\{\epsilon\}\cup\{X,Y,Z,\ldots\}$是一个 \emph{进程常量(process constants)}的集合。
$S=\{\alpha_1,\alpha_2,\ldots\}$被称为\emph{进程项(process terms)}的集合,它被用来刻画系统的状态,可以由一下的BNF产生:
$$\alpha\Coloneqq \epsilon\mid X\mid \alpha_1.\alpha_2\mid \alpha_1\para \alpha_2$$
其中
\begin{itemize}
\item $\epsilon$被称为\emph{空进程(empty process)};
\item $\alpha_1.\alpha_2$是一个\emph{串行(sequential)}进程;
\item $\alpha_1\para \alpha_2$是一个\emph{并行(parallel)}进程。
\end{itemize}
我们这里用小写希腊字母$\alpha,\beta,\gamma,\ldots$来表示进程项。
\end{defn}

有了进程项的定义,对于一个进程演算系统,就定义它的\emph{操作语义(Operational Semantics)}。这里,我们使用\emph{标号迁移系统(Labeled Transition System 简称LTS)}来定义PRS中的模型所遵循的语义规则。

\begin{defn}[标号迁移系统 LTS]\label{def:lts}
一个\emph{标号迁移系统(LTS)}是一个五元组$(S,Act,\longrightarrow,\alpha_0,F)$,其中
\begin{itemize}
    \item $S$是一个\emph{状态(states)}的有限集合,
    \item $Act$是一个\emph{标号(labels)}的有限集合,
    \item $\longrightarrow\;\subseteq S\times Act\times S$是一个\emph{转换关系(transition relation)},
    \item $\alpha_0\in S$是一个给定的\emph{初始状态(start state)},
    \item $F\subseteq S$是一个\emph{终结状态(final states)}的有限集合,这意味着对于任何$\alpha\in F$不存在$a\in Act$和$\beta\in S$使得$\alpha\stackrel{a}{\longrightarrow}\beta$。
\end{itemize}
我们通常将$(\alpha,a,\beta)\in \longrightarrow$记做$\alpha\stackrel{a}{\longrightarrow}\beta$。
\end{defn}

接下来就可以利用\emph{语义推导规则(inference rules)}来得到PRS模型的操作语义。LTS所定义的语义转换关系是由形如$\alpha\act{a}\beta$的规则的所构成的有限集合$\Delta$所生成的。对于任意$a\in Act$,语义迁移关系$\act{a}$是从以下的语义推导规则构造的最小的转换关系:
$$\begin{tabular}{cc}
$\inference{\alpha\stackrel{a}{\longrightarrow}\beta\in\Delta}{\alpha\stackrel{a}{\longrightarrow}\beta}$&
$\inference{\alpha\stackrel{a}{\longrightarrow}\alpha'}{\alpha.\beta\stackrel{a}{\longrightarrow}\alpha'.\beta}$ \\
\\
$\inference{\alpha\stackrel{a}{\longrightarrow}\alpha'}{\alpha\para \beta\stackrel{a}{\longrightarrow}\alpha'\para \beta}$ & $\inference{\beta\stackrel{a}{\longrightarrow}\beta'}{\alpha\para \beta\stackrel{a}{\longrightarrow}\alpha\para \beta'}$
\end{tabular}$$

\subsection{层次结构}
\label{subsec:hierarchy}

PRS利用对语义规则中进程项类型的分类,分成了几个子模型。这些子模型大多可以和一些常用的进程模型所对应,它们之间有着一个关于互模拟关系表达能力包含关系的层次结构。这里我们首先先根据连结符,将进程项分为四类:
\begin{enumerate}
    \item $\mathbf{1}$: 仅仅由\emph{单个(single)}进程常量构成的项,形如$X$。
    \item $\mathbf{S}$: 单个进程常量或者\emph{串行连结(sequential composition)}的进程常量构成的项,形如$X.Y.Z$。
    \item $\mathbf{P}$: 单个进程常量或者\emph{并行连结(parallel composition)}的进程常量构成的项,形如$X\para Y\para Z$.
    \item $\mathbf{G}$: 由任意串行或者并行连结的进程常量构成的项,如 $(X.(Y\para Z))\para W$。
\end{enumerate}
下面我们将给出PRS的严格定义:
\begin{defn}[进程重写系统 PRS]\label{def:prs}
令$\Xi,\Pi\in\{\mathbf{1},\mathbf{S},\mathbf{P},\mathbf{G}\}$.
一个\emph{$(\Xi,\Pi)$-PRS}是一个满足如下条件的规则集合$\Delta$:
对于每条规则$\alpha \stackrel{a}{\longrightarrow} \beta\in \Delta$有
\begin{itemize}
	\item $\alpha\in\Xi\setminus \{\epsilon\}$,
	\item $\beta\in\Pi$,
	\item 系统的初始状态由一个进程项$\alpha_0\in\Xi$给定。
\end{itemize}
一个 $(\mathbf{G},\mathbf{G})$-PRS即为一个一般的PRS.
\end{defn}
不失一般性,本文假定PRS的初始状态$\alpha_0$为一个单独常量进程项。

分类了PRS中各种模型之后,我们可以得到图\ref{fig:prs}中的\emph{层次结构(hierarchy)}。接下来,将介绍本文研究中所涉及到的一些子模型。
\begin{figure}
\begin{center}
\begin{tabular}{c}
\xymatrix{
& \mathsf{PRS}\; (\mathbf{G},\mathbf{G})\ar@{-}[dl] \ar@{-}[dr] & \\
\mathsf{PAD}\; (\mathbf{S},\mathbf{G}) \ar@{-}[d] \ar@{-}[dr] &  & \mathsf{PAN}\; (\mathbf{P},\mathbf{G}) \ar@{-}[d] \ar@{-}[dl]\\
\mathsf{PDA}\; (\mathbf{S},\mathbf{S}) \ar@{-}[d] & \mathsf{PA}\; (\mathbf{1},\mathbf{G})\ar@{-}[dl] \ar@{-}[dr] & \mathsf{PN}\; (\mathbf{P},\mathbf{P}) \ar@{-}[d]\\
\mathsf{BPA}\; (\mathbf{1},\mathbf{S}) \ar@{-}[dr]&  & \mathsf{BPP}\; (\mathbf{1},\mathbf{P}) \ar@{-}[dl]\\
 & \mathsf{FS}\; (\mathbf{1},\mathbf{1}) & \\
}
\end{tabular}
\end{center}
\caption{\textsf{PRS}层次结构 PRS-Hierarchy}
\label{fig:prs}
\end{figure}

\subsection{PRS中的模型}
\label{subsec:models}


\section{互模拟等价关系}
\label{sec:bis}

\subsection{强互模拟关系}
\label{subsec:strong-bis}

\subsection{考虑内部动作的互模拟关系}
\label{subsec:silent-bis}

\section{无限状态系统验证问题}
\label{sec:verfication}

\subsection{几个问题}
\label{subsec:prob}

\subsection{研究目标}
\label{subsec:goal}
